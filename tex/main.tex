\documentclass[12pt, letterpaper, preprint]{aastex}
\usepackage[breaklinks,colorlinks, urlcolor=blue,citecolor=blue,linkcolor=blue]{hyperref}
\usepackage{hyperref}
\usepackage{color}
%%% This file is generated by the Makefile.
\newcommand{\giturl}{\url{https://github.com/changhoonhahn/centralMS}}
\newcommand{\githash}{d9ea498}\newcommand{\gitdate}{2019-05-13}\newcommand{\gitauthor}{changhoonhahn}


% typesetting shih
\linespread{1.08} % close to 10/13 spacing
\setlength{\parindent}{1.08\baselineskip} % Bringhurst
\setlength{\parskip}{0ex}
\let\oldbibliography\thebibliography % killin' me.
\renewcommand{\thebibliography}[1]{%
  \oldbibliography{#1}%
  \setlength{\itemsep}{0pt}%
  \setlength{\parsep}{0pt}%
  \setlength{\parskip}{0pt}%
  \setlength{\bibsep}{0ex}
  \raggedright
}
\setlength{\footnotesep}{0ex} % seriously?

\newcommand\tab[1][1cm]{\hspace*{#1}}
\newcommand{\todo}[1]{{\bf \textcolor{red}{#1}}}
\newcommand{\beq}{\begin{equation}}
\newcommand{\eeq}{\end{equation}}
\newcommand{\overbar}[1]{\mkern 1.5mu\overline{\mkern-1.5mu#1\mkern-1.5mu}\mkern 1.5mu}
\newcommand{\avgSFR}{\overline{\raisebox{0pt}[1.2\height]{SFR}}}
\newcommand{\SFR}{\mathrm{SFR}}
\newcommand{\fq}{f_\mathrm{Q}}
\newcommand{\fqcen}{f_\mathrm{Q}^\mathrm{cen}}
\newcommand{\zinit}{z_\mathrm{initial}}
\newcommand{\taucen}{\tau_\mathrm{Q}^\mathrm{cen}}
\newcommand{\logsfr}{\log \, \mathrm{SFR}}
\newcommand{\bitem}{\begin{itemize}}
\newcommand{\eitem}{\end{itemize}}
\newcommand{\musfms}{\log\,\overline{\mathrm{SFR}}_\mathrm{MS}}

\begin{document}\sloppy\sloppypar\frenchspacing

\title{Star Formation Main Sequence in a Hierarchical Universe} 
%\title{Central Galaxies on the Main Sequence} 
\date{\texttt{DRAFT~---~\githash~---~\gitdate~---~NOT READY FOR DISTRIBUTION}}
\author{ChangHoon~Hahn\altaffilmark{1}, 
Jeremy L.~Tinker\altaffilmark{2}, 
Andrew R.~Wetzel\altaffilmark{3,4,5}}
\altaffiltext{1}{Lawrence Berkeley National Laboratory, 1 Cyclotron Road, Berkeley, CA 94720}
\altaffiltext{2}{Center for Cosmology and Particle Physics, Department of Physics, New York University, 4 Washington Place, New York, NY 10003}
\altaffiltext{3}{TAPIR, California Institute of Technology, Pasadena, CA USA}
\altaffiltext{4}{Carnegie Observatories, Pasadena, CA USA}
\altaffiltext{5}{Department of Physics, University of California, Davis, CA USA}
\email{changhoon.hahn@lbl.gov}

\begin{abstract}
    \todo{motivation, methodology, impact.}
    In observations star forming galaxies form a tight $log\;M_*$ to $log\;SFR$ 
    relation referred to as the {\em star formation main sequence} (SFMS) out to $z\sim2$. 
    Beyond the evolution ``along'' this SFMS, however, the star formation histories of star 
    forming galaxies have not been precisely characterized. 
    The SFH of these galaxies govern SMF, SFMS, and also observed constraints on the stellar mass to halo mass
    relation. 

    By combining high-resolution cosmological $N$-body simulation with observed evolutionary 
    trends of SF galaxies, we construct a model that tracks the evolution of star forming 
    central galaxies over the redshift $z < 1$. Comparing this model 

    Observations find a remarkably small scatter in the stellar mass to halo mass relation. 
    Somehow the star formation histories of galaxies must 
    
    According to observations, star forming galaxies form a tight $log\;M_*$ to $log\;SFR$ 
    relation referred to as the ``star formation main sequence'' out to $z\sim2$. 
\end{abstract}
\keywords{methods: numerical -- galaxies: clusters: general -- 
galaxies: groups: general -- galaxies: evolution -- galaxies: haloes -- 
galaxies: star formation -- cosmology: observations.}

\section{Introduction}
\bitem 
\item Motivate why we think SF galaxies evolve along the main sequence  
\item Discuss the current thought process on galaxy assembly bias 
\item Explain the limitation of SFH derivable from observations (Claire's fisher matrix paper would be really good; ask her about the details) 
\item Observations also can't provide detail host dark matter halo properties
\item So the approach with combining observations with N-body (empirical modeling) is very effective in the context of the halo.
\item Maybe talk about how the bigger context of why this is important?  
\item Why only centrals -- because our current best understanding of satellites is that they quench after infall, so it doesn't make sense to look at them
\item our model goes from $z < 1$ because beyond that the observations are statistically meaningless.  
\eitem 

%%%%%%%%%%%%%%%%%%%%%%%%%%%%%%%%%%%%%
% Figure 1 
%%%%%%%%%%%%%%%%%%%%%%%%%%%%%%%%%%%%%
\begin{figure}
\begin{center}
\includegraphics[width=0.9\textwidth]{figs/groupcat.pdf}
    \caption{Central galaxies of the SDSS DR7 group catalog. \emph{Left:} We plot 
    the SFR-$M_*$ relation of the SDSS central galaxies. The contours illustrate the bimodal
    distribution of the galaxy properties and mark the star-forming and quiescent populations. 
    The transitioning galaxies lie on the ``green'' valley between the star-fomring and quiescent
    modes. %The dashed line represents the linear fit to the SFMS as described in Section~\ref{sec:sfcen}. 
    \emph{Right:} We plot the distribution of $\mathrm{log}\,\mathrm{SSFR}$ for SDSS centrals
    with $10.6 < \mathrm{log}\,M_* < 10.8$. Shaded in blue, we plot the SFMS component of our 
    GMM fit of the SFR-$M_*$ relation described in Section~\ref{sec:sfcen}. Based on this fit, 
    galaxies in the SFMS account for approximately $f_\mathrm{SFMS} = 0.21$ of the central 
    galaxies in the stellar mass bin.} \label{fig:groupcat}
\end{center}
\end{figure}
%%%%%%%%%%%%%%%%%%%%%%%%%%%%%%%%%%%%%

\section{Central Galaxies of SDSS DR7} \label{sec:sdss}
We construct our galaxy sample following the sample selection of \cite{tinker2011}. 
We select a volume-limited sample of galaxies with $M_r −5 log(h) < −18$ and complete in
$M_* > 10^{9.4} M_\odot$ from the NYU Value-Added Galaxy Catalog \citep[VAGC;][]{blanton2005}
of the Sloan Digital Sky Survey Data Release 7~\citep[SDSS DR7;][]{abazajian2009} at 
$z \approx 0.04$. The stellar masses of these galaxies are estimated using the
$\mathtt{kcorrect}$ code~\citep{blanton2007} assuming a~\cite{chabrier2003} initial
mass function. The star formation of the galaxies are estimated spectroscopically using the
specific star formation rates (SSFR) from the current release of the MPA-JHU spectral 
reductions\footnote{http://wwwmpa.mpa-garching.mpg.de/SDSS/DR7/}~\citep{brinchmann2004}.
Generally speaking, $\mathrm{SSFR} > 10^{-11}\mathrm{yr}^{-1}$ are derived from 
$\mathrm{H}_\alpha$ emission, $10^{-11} > \mathrm{SSFR} > 10^{-12}\mathrm{yr}^{-1}$
are derived from a combination of emission lines, and $\mathrm{SSFR} < 10^{-12}\mathrm{yr}^{-1}$
are based on $D_n 4000$~\citep[see discussion in][]{wetzel2013}. We note that 
$\mathrm{SSFR} < 10^{-12}\mathrm{yr}^{-1}$ should only be considered upper limits 
to the true galaxy SSFR~\citep{salim2007}.

From our galaxy sample, we identify the central galaxies using the \cite{tinker2011} halo-based 
group-finding algorithm, which is based on the~\cite{yang2005} algorithm and tested 
in~\cite{campbell2015}. The algorithm assigns a probability of being a satellite,
$P_\mathrm{sat}$, to each galaxy in the sample. Galaxies with $P_\mathrm{sat} \geq 0.5$ 
are classified as satellites and $P_\mathrm{sat} < 0.5$ are classified as centrals. 
In this paper we focus on central galaxies. With any group finding algorithm, galaxies are 
misassigned due to projection effects and redshift space distortions. The purity 
of the full central galaxy sample is $\sim 90\%$ with a completeness of $\sim 95\%$~\citep{tinker2017}.
Furthermore, \cite{campbell2015} find that the algorithm robustly identifies red and blue centrals
as a function of stellar mass, which is highly relevant to our analysis.  

In the left panel of Figure~\ref{fig:groupcat}, we plot the SFR-$M_*$ distribution of
the SDSS DR7 central galaxies. In the right panel, we plot the distribution of SSFR, 
$p(\log \mathrm{SSFR})$, for galaxies with $10.6 < \log \,M_* < 10.8$ (stellar mass range 
highlighted on the left panel). Both panels of Figure~\ref{fig:groupcat} illustrate the 
bimodality in the galaxy sample. The SFR-$M_*$ distribution also illustrate the correlation
between SFR and $M_*$ in star-forming galaxies \emph{i.e.} the star-formation main sequence 
(SFMS).

\section{Model: Simulated Central Galaxies} \label{sec:sim}
We're interesting in constructing a model that tracks central galaxies and 
their star formation within the heirarchical growth of their host halos. This 
requires a cosmological $N$-body simulation that accounts for the complex 
dynamical processes that govern the host halos of galaxies. In this paper 
we use the high resolution $N$-body simulation from~\cite{wetzel2013} generated 
using the \cite{white2002} $\mathtt{TreePM}$ code with flat $\Lambda$CDM cosmology 
($\Omega_m =0.274, \Omega_b = 0.0457, h = 0.7, n=0.95, \mathrm{and} \sigma_8 = 0.8$).
From initial conditions at $z = 150$ generated from second-order Lagrangian 
Perturbation Theory, $2048^3$ particles with mass of $1.98 \times 10^8\,M_\odot$ are 
evolved in a $250 \mathrm{Mpc}/h$ box with a Plummer equivalent smoothing of 
$2.5\,\mathrm{kpc}/h$. For a more detailed description of the simulation, we 
refer readers to~\cite{wetzel2013, wetzel2014}.

From the $\mathrm{TreePM}$ $N$-body simulation, `host halos' are identified 
using the Friends-of-Friends (FoF) algorithm of \cite{davis1985} with 
linking length of $b = 0.168$ times the mean inter-partcile spacing. Within 
these host halos, \cite{wetzel2013} identifies `subhalos' as overdensities 
in phase space through a six-dimensional FoF algorithm~\citep[FoF6D][]{white2010}. 
The host halos and subhalos are then tracked across the $45$ simulation 
outputs from $z = 10$ to $0$ to build merger trees~\citep{wetzel2009,wetzel2010}. 
The most massive subhalos in newly-formed host halos at a given simulation 
output are defined as the `central' subhalo. A central subhalo retains its 
`central' definition until it falls into a more massive host halo, at which 
point it becomes a `satellite' subhalo. 

Each subhalo is assigned a $M_\mathrm{peak}$, the maxmum host halo mass that 
it ever had as a central subhalo. Using $M_\mathrm{peak}$, we construct a galaxy 
catalog from the subhalos using subhalo abundance 
matching~\citep[SHAM;][]{conroy2006,vale2006,yang2009,wetzel2012,leja2013,wetzel2013,wetzel2014,hahn2017a}. 
In principle, SHAM assumes a one-to-one mapping between subhalo 
$M_\mathrm{peak}$ and galaxy stellar mass $M_*$: $n(> M_\mathrm{peak}) > n(> M_*)$
that preserves the rank ordering. In practice, we apply a $0.2$ dex log-normal 
scatter in $M_∗$ at fixed $M_\mathrm{peak}$ based on observations of the stellar 
mass to halo mass relation (SMHMR; \todo{bunch of SMHMR citations}). \cite{gu2016} 
compile empirical constraints on the scatter of this stellar mass to halo 
mass relation ($\sigma_{\log M_*}$). Using the SHAM mapping, we can 
assign galaxy stellar mass to subhalos based on observed stellar mass 
functions (SMFs) at the redshifts of the simulation outputs (snapshots). 

We use the SMF from \cite{li2009} at $z = 0.05$ and at higher redshifts 
interpolate between the \cite{li2009} SMF and the SMF from \cite{marchesini2009} 
at $z = 1.6$. We choose the \cite{li2009} SMF because it is based on the 
same SDSS NYU-VAGC sample as our SDSS DR7 group catalog (Section~\ref{sec:sdss}). 
We choose the \cite{marchesini2009} SMF, amongst others, because it produces 
interpolated SMFs that monotonically increase at $z < 1$. As noted in 
\cite{hahn2017a}, at $z \approx 1$, the SMF interpolated between the 
\cite{li2009} and \cite{marchesini2009} SMFs is consistent with more 
recent measurements from \cite{muzzin2013} and \cite{ilbert2013}. 
At each snapshot, we independently use SHAM to assign galaxy $M_*$. 
This way, we not only track the evolution of subhalos, but also the 
the galaxies' $M_*$. With the $45$ snapshots outputs from our simulation, 
we can in principle track the central galaxies back to $z \sim 10$. 
However, we restrict ourselves to snapshots at $z \lesssim 1$, where we 
have the most statistically meaningful observations. We next describe
how we select star forming central galaxies in our model and initalize them. 
%As we desribe later in this section, however, we're only interested in the snapshots at $z \approx 0.05$ and $1$, respectively. 
\todo{TBD: Perhaps mention in appendix how we test different SMF assumptions}% (In Section~\ref{app:z1},)

%%%%%%%%%%%%%%%%%%%%%%%%%%%%%%%%%%%%%
% Figure 2 
%%%%%%%%%%%%%%%%%%%%%%%%%%%%%%%%%%%%%
%\begin{figure}
%\begin{center}
%\includegraphics[width=0.5\textwidth]{figs/fq_fsfms.pdf}
%\caption{SFMS fraction versus quiescent fraction from Hahn}
%\label{fig:fq_fsfms}
%\end{center}
%\end{figure}
%%%%%%%%%%%%%%%%%%%%%%%%%%%%%%%%%%%%%

\subsection{Selecting Star Forming Central Galaxies}  \label{sec:sfcen}
In our model, we're interested in tracking the evolution of the
SFR and stellar mass of SF central galaxies. To construct such a model, 
we first need to select star-forming galaxies from the central galaxies 
in our simulation described earlier in this section. Since we later 
compare our model to observation, our selection is based on 
$f^\mathrm{cen}_\mathrm{SFMS}(M_*)$, the fraction of central galaxies 
within the star-forming main sequence, measured from the SDSS DR7 VAGC 
(Section~\ref{sec:sdss}). Below, we describe how we derive this 
$f^\mathrm{cen}_\mathrm{SFMS}(M_*)$ and use it to select star-forming 
central galaxies in our model. Afterwards we describe how we initalize 
the SFRs and $M_*$ of these galaxies in our model.

Often in the literature, an empirical color-color or SFR-$M_*$ cut 
that separates the two main modes (red/blue or star-forming/quiescent) 
in the distribution is chosen to classify 
galaxies~\citep[\emph{e.g.}][]{baldry2006, blanton2009, drory2009, peng2010, moustakas2013, hahn2015}.
The red/quiescent or blue/star-forming fractions derived from this sort of 
classification, by construction, depend on the choice of cut and neglect the 
transitioning galaxies~\emph{i.e.} the galaxies in the ``green valley''. 
For our $f^\mathrm{cen}_\mathrm{SFMS}(M_*)$ measurement, we instead use a
method from Hahn et al.~(in prep) and Tjitske et al.~(in prep), which is 
based on fitting the SFMS from the SFR-$M_*$ distribution 

The SFMS fitting scheme first divides the SDSS DR7 VAGC central 
galaxy sample (Section~\ref{sec:sdss}) into stellar mass bins of 
width $\Delta \log M_* = 0.2~\mathrm{dex}$. We then fit the SSFR 
distribution of each bin using Gaussian mixture models (GMMs) with 
$1 - 3$ components using the expectation-maximization 
algorithm~\citep[EM;][]{dempster1977, neal1998}. We restrict 
ourselves to models with a maximum of $3$ components for the three 
possible galaxy classifications: quiescent, star-forming, and green 
valley populations. From the three GMMs, we select the model with 
the lowest Bayesian Information Criteria~\citep[BIC;][]{schwarz1978}. 
The Gaussian component of this GMM with mean $\log \mathrm{SSFR} > -11$ 
is identified as the SFMS. In the rare cases when more than one GMM 
component has mean $\log \mathrm{SSFR} > -11$, we compare the weights 
of the components.  If the weight of one component is less than a 
third of the other, we take the component with the higher weight to 
represent the SFMS. Otherwise, we omit the stellar mass bin altogether. 
\todo{sentence about how for the VAGC, we don't omit any stellar mass bins} 
The weight of the SFMS GMM component provides an estimate of 
$f^\mathrm{cen}_\mathrm{SFMS}$ for the given stellar mass bin. 
In the right panel of Figure~\ref{fig:groupcat}, we plot the SFMS 
GMM component (blue shaded region) of the $p(\log \mathrm{SSFR})$ 
for the SDSS DR7 central galaxies within $10.6 < \log M_* < 10.8$. 
The SFMS constitutes $f^\mathrm{cen}_\mathrm{SFMS} = 0.21$ of the 
SDSS central galaxies in this stellar mass bin. 

Next, with the $f^\mathrm{cen}_\mathrm{SFMS}$ values spanning 
the different stellar mass bins, we fit $f^\mathrm{cen}_\mathrm{SFMS}$ as 
a function of $\log\,M_*$. Using a fiducial stellar mass of 
$\log\,M_\mathrm{fid} = 10.5$, we derive the following best-fit 
\beq \label{eq:f_cen_sfms}
f^\mathrm{cen}_\mathrm{SFMS, bestfit}(M_*) = -0.627\,(\log\,M_* - 10.5) + 0.354. 
\eeq
We note that this $f^\mathrm{cen}_\mathrm{SFMS, bestfit}(M_*)$ is 
in good agreement with the $f_\mathrm{Q}^\mathrm{cen}(M_*; z\sim0)$ 
fit from \cite{hahn2017a}. For each central galaxy in our simulation, 
we assign a probability of it being on the SFMS, using 
$f^\mathrm{cen}_\mathrm{SFMS, bestfit}(M_*)$ with $M_*$ at $z \sim 0$ 
assigned through SHAM. Based on these probablities, we randomly 
identify central galaxies from our simulation as star-forming at 
$z \sim 0$.

At this point, to initialize our SF centrals,  we make the assumption 
that once a SF galaxy quenches its star formation, it remains quiescent. 
\todo{maybe something about rejuvenation?} 
Without any quenched galaxies rejuvenating their star formation, the central galaxies that 
we place on the SFMS at $z \sim 0$ were also on the SFMS at $z \sim 1$ --- 
the initial redshift of our model. At $z \sim 1$, we initialize the SF centrals
with SHAM $M_*$s and assign their initial SFRs based on the observed SFR-$M_*$ 
relation of the SFMS. Observations in the literature at these redshifts, 
however, not only use galaxy properties derived differently from the SDSS VAGC
but they also find SFMS with significant discrepancies from one another. 
In \cite{speagle2014}, they compile the SFR-$M_*$ relation of the SFMS 
from 25 studies in the literature, each with different methods of deriving 
galaxy properties. Even after the calibration, for a fixed $M_* = 10^{10.5}\, M_\odot$, 
the SFRs of the SFMS from different analyses at $z \sim 1$ vary by more than a factor of 
2~\citep[see Figure 2 of][]{speagle2014}. This is particularly concerning 
since the redshift evolution of the SFMS dictate the SFR and $M_*$ evolution 
of SF centrals in our model. So to deal with this uncertainty, we parameterize 
the SFMS SFR ($\log\,\mathrm{SFR}_\mathrm{MS}(M_*, z)$) with free parameters 
that characterize the redshift dependence and marginalize over them later in 
the analysis. 

For our $\log\,\mathrm{SFR}_\mathrm{MS}(M_*, z)$, at $z \sim 0$ we anchor our 
SFMS to the SFMS of the SDSS central galaxies, which we obtain from the 
SFMS GMM components used to estimate $f^\mathrm{cen}_\mathrm{SFMS}$. Using 
the mean $\log\, \mathrm{SFR}$ of the SFMS GMM component at each stellar 
mass bin, we linearly fit $\log\, \mathrm{SFR}(M_*)$ (see dashed line in 
Figure~\ref{fig:groupcat}). Then for $z > 0$, we include two free 
parameters to account for the redshift evolution of the amplitude and 
slope of the SFMS SFR-$M_*$ relation: $A_z$ and $m_z$, respectively. 
We parameterize the mean $\log\,\mathrm{SFR}$ of the SFMS as, 
\beq \label{eq:logsfr_ms}
\log\,\overline{\mathrm{SFR}}_\mathrm{MS}(M_*, z) = . 
\eeq
Then we initialize the $z \sim 1$ SFRs of our SF centrals by randomly sampling 
a log-normal distribution centered about $\log\,\overline{\mathrm{SFR}}_\mathrm{MS}(M_*, z=1)$ 
with a constant scatter of $0.3\,\mathrm{dex}$, motivated from 
observations~\citep{noeske2007, elbaz2007, daddi2007, whitaker2012}. 
Later in our analysis, when we choose priors for $A_z$ and $m_z$, we conservatively 
choose a range that encompass the best-fit SFMS from~\cite{speagle2014} and measurements from~\cite{moustakas2013} 
and~\cite{lee2015}. With our SF centrals initalized at $z \sim 1$, next, we 
describe how we evolve their SFR and $M_*$ in our model.

%%%%%%%%%%%%%%%%%%%%%%%%%%%%%%%%%%%%%
% Figure 3 
%%%%%%%%%%%%%%%%%%%%%%%%%%%%%%%%%%%%%
\begin{figure}
\begin{center}
\includegraphics[width=0.9\textwidth]{figs/sfh_pedagogical.pdf}
    \caption{Pedagogical illustration of how our model, described in Sections~\ref{sec:sfcen} 
    and~\ref{sec:modelevol}, evolves star forming central galaxies.
    \emph{Left panel}: . 
    \emph{Right panel}:}
\label{fig:sfh_model}
\end{center}
\end{figure}
%%%%%%%%%%%%%%%%%%%%%%%%%%%%%%%%%%%%%

%%%%%%%%%%%%%%%%%%%%%%%%%%%%%%%%%%%%%
% Figure 4 
%%%%%%%%%%%%%%%%%%%%%%%%%%%%%%%%%%%%%
\begin{figure}
\begin{center}
\includegraphics[width=0.6\textwidth]{figs/illustris_sfh.pdf} 
    \caption{Star formation rate with respect to the mean SFMS ($\Delta \logsfr$) as 
    a function of cosmic time for star-forming galaxies in the Illustris simulation. 
    These galaxies have stellar masses within the range $10^{10.5}-10^{10.6}M_\odot$ 
    at $z\sim0$. $\musfms$ is fit using the same as the SFMS fits for the SDSS 
    centrals in Section~\ref{sec:something}. The $\Delta \logsfr(t)$ of these Illustris 
    star-forming galaxies stochastically fluctuate about the mean SFMS. Our prescription 
    for the SFRs of our SF centrals in Eq.\ref{eq:something} is modeled to resemble this
    $\Delta \logsfr(t)$ behavior.}
\label{fig:illsfh}
\end{center}
\end{figure}
%%%%%%%%%%%%%%%%%%%%%%%%%%%%%%%%%%%%%

%%%%%%%%%%%%%%%%%%%%%%%%%%%%%%%%%%%%%
% Figure 5 
%%%%%%%%%%%%%%%%%%%%%%%%%%%%%%%%%%%%%
\begin{figure}
\begin{center}
\includegraphics[width=0.9\textwidth]{figs/qaplot_abc.pdf}
\caption{}
\label{fig:abc_demo}
\end{center}
\end{figure}
%%%%%%%%%%%%%%%%%%%%%%%%%%%%%%%%%%%%%

\subsection{Evolving along the Main Sequence} \label{sec:modelevol} 
The tight correlation between the SFRs and stellar masses of star-forming 
galaxies (the so-called SFMS) spans over four orders of magnitude in stellar 
mass and extends beyond the local universe out to $z > 2$ 
(\emph{e.g.}~\citealt{noeske2007,daddi2007,elbaz2007,salim2007,santini2009,karim2011,whitaker2012,moustakas2013,lee2015}; see also references in \citealt{speagle2014}). 
In bins of $\mathrm{log}\,M_*$, the SFRs of galaxies on the SFMS are observed
to follow a log-normal distribution with a roughly constant scatter of 
$0.3\,\mathrm{dex}$. Given its persistence for star-forming galaxies in 
the local Universe, the SFMS provides a straightforward relationship to 
characterize the SFRs and $M_*$s of star-forming galaxies 
throughout $z < 1$. More specifically, we can characterize the star formation
histories (SFHs) of our star-forming centrals with respect to the  mean 
$\logsfr$ of the SFMS (Eq.~\ref{eq:logsfr_ms}):
\beq \label{eq:logsfr_sf} 
\logsfr(M_*, t) = \log\,\overline{\mathrm{SFR}}_\mathrm{MS}(M_*, t) + \Delta \logsfr.
\eeq
Since SFH determines the $M_*$ growth of a galaxy, in our model, $\Delta \logsfr$
dictates both the SFHs and $M_*$ evolution of the star-forming centrals in our 
model. Below, we describe our specific prescription for $\Delta \logsfr$.

In the previous section, we selected the initial SFRs of our star-forming
centrals by sampling the log-normal distribution about the mean SFMS at 
$z \sim 1$. The most naive prescription for $\Delta \logsfr$, would be to 
fix $\Delta \logsfr$ based on the initial offsets from the mean. SF centrals 
with higher than average initial SFRs continue evolving above the average SFMS, 
while SF centrals with lower than average initial SFRs continue evolving below 
the average SFRs. With such SFHs, as we demonstrate later in detail, the scatter 
in stellar mass of the SF centrals at fixed halo mass diverges substantially 
with cosmic time. Furthermore, when we examine the SFHs of star-forming galaxies
of the Illustris simulation~\cite{vogelsberger2014,genel2014}, we find little
motivation for such prescription. In Figure~\ref{fig:illsfh}, we plot 
$\Delta \logsfr$ of star-forming Illustris galaxies as a function of cosmic time. 
At $z = 0$, these galaxies have stellar masses within $10^{10.5}-10^{10.6}M_\odot$. 
When calculating $\Delta \logsfr$, we determine the mean SFMS  at every 
simulation output using the same fitting method we used for the SDSS centrals 
in Section~\ref{sec:sfcen}. Rather than staying fixed, Figure~\ref{fig:illsfh} 
illustrates that the $\Delta \logsfr$s of the Illustris galaxies stochastically 
vary about SFMS. 

Motivated by the Illustris SFHs, we introduce the ``\emph{star formation 
duty cycle}'' to our prescription of $\Delta \logsfr$. We parameterize 
$\Delta \logsfr$ to randomly flucutate about the mean SFMS on some duty cycle 
timescale $t_\mathrm{duty}$. In the left panel of Figure~\ref{fig:sfh_model}, we 
present the time evolution of $\Delta \logsfr$ for two pedagogical SFHs of SF 
centrals in our model with $t_\mathrm{duty} = 0.5$ Gyr (blue) and $5$ Gyr (orange).  
The shaded region represents the $0.3\,\mathrm{dex}$ $1\sigma$ log normal 
scatter of the SFMS SFR.  
\todo{A brief comment about how we obviously do not expect such a simplified 
prescription for SFH to precisely reflect reality. However, we expect it to 
loosely encapsulate the stochasticity of star formation from accretion and star 
bursts.} Furthermore, this prescription of $\Delta \logsfr$ produces SFHs that 
reproduce the log-normal SFR distribution of the SFMS at any point in the model. 

Using our $\Delta \logsfr$ prescription, we can now evolve both the 
SFR and $M_*$ of our SF centrals along the main sequence. As Eq.~\ref{eq:logsfr_sf}
reveals, SFR of our SF centrals is function of $M_*$. Meanwhile, $M_*$ is the
integral of the SFR over time: 
\beq \label{eq:integ_mass} 
M_*(t) = f_\mathrm{retain} \int\limits_{t_0}^{t} \mathrm{SFR(M_*, t)}\,\mathrm{d}t + M_0. 
\eeq
$f_\mathrm{retain}$ here is the fraction of stellar mass that is retained after 
supernovae and stellar winds; we use $f_\mathrm{retain} = 0.6$~\citep{wetzel2013}. 
$t_0$ and $M_0$ are the initial cosmic time and stellar mass at $z \sim 1$, respectively. 
Combining Eqs.~\ref{eq:logsfr_sf} and~\ref{eq:integ_mass}, we get a different 
equation, which we solve to evolve the SFR and $M_*$ of our SF centrals.  
The right panel of Figure~\ref{fig:sfh_model} presents the SFR and $M_*$ 
evolutions of the two pedagogical SF centrals from the left panel. The dotted 
lines represent $\log\,\overline{\mathrm{SFR}}_\mathrm{MS}$ of the SFMS at different 
redshifts starting from $z = 1$ to $0.05$. %We also $\log\,\overline{\mathrm{SFR}}_\mathrm{MS}$

Now that we have a model that evolves the SFR and $M_*$ of the SF centrals, 
we can compare our model to observations. Our model is constructed using the 
SMF and SMHMR at $z = 1$ and based on the observed evolution of the SFMS. 
It has two free parameters $A_z$ and $m_z$, which allow for a flexible redshift 
dependence of the SFMS. At the final snapshot, $z \sim 0$, our model reproduces 
the observed SFR distribution by construction. We also want our model to reproduce 
the SDSS DR7 VAGC $z \sim 0$~\cite{li2009} SMF. To compare our model to observation
we use the parameter estimation framework of Approximate Bayesian Computation (ABC). 
ABC has the advantage over standard approaches to parameter inference in that it does not 
require evaluating the likelihood. It relies only on a simulation of the observed 
data and a distance metric to quantify the ``closeness'' between the observed data
and simulation. Many variations of ABC has been used in astronomy and 
cosmology~\citep[\emph{e.g.}][]{cameron2012,weyant2013,ishida2015,alsing2018}; 
we use ABC in conjunction with the efficient Population Monte Carlo (PMC)
importance sampling as in~\citep{hahn2016, hahn2017}. For the priors of $A_z$ and $m_z$, 
we uniform distributions spanning the ranges \todo{numbers} and \todo{numbers}, 
respectively. As we mentioned in Section~\ref{sec:sfcen}, the range of the prior 
were conservatively chosen to encompass the best-fit SFMS from~\cite{speagle2014} 
and measurements from~\cite{moustakas2013} and~\cite{lee2015} at $z \sim 1$. 
Finally, to compare the SMF of our model to the SDSS central galaxy SMF, we use the following
distance metric: 
\beq
\rho_\Phi = \sum\limits_{M} \left( \frac{\Phi^\mathrm{sim}(M) - f^\mathrm{cen}(M) \Phi^\mathrm{SDSS}(M)}{\sigma_\Phi(M)}\right)^2.
\eeq
$\Phi^\mathrm{sim}(M)$ is the SMF of our model, $\Phi^\mathrm{SDSS}(M)$ is 
the~\cite{li2009} SDSS DR7 SMF, and $\sigma_\Phi(M)$ is the SMF uncertainty 
derived using mock catalogs from~\cite{li2009}. For the rest of our ABC-PMC 
implementation, we strictly follow the implementation of \cite{hahn2017b} 
and~\cite{hahn2017a}. We refer reader to those papers for further details.

In Figure~\ref{fig:abc_demo}, we present the SMFs (left), SFMSs (center), and 
$\sigma_{\log\,M_*}(\mathrm{log}\,M_\mathrm{halo}$ (right) for the ABC outputs 
of two different SFR prescriptions, one with no duty cycle (red) and the other 
with $t_\mathrm{duty} = 1\,\mathrm{Gyr}$. The left panel illustrates that the 
ABC output of our model successfully reproduces the $z \sim 0$ SMF of SDSS 
centrals. The center panel illustrates
Lastly, the right panel demonstrates that different SFR prescriptions for  
SF galaxies produce different uncertainty in the SMHMR relation. Since we have
observed constaints on the uncertainty of the SMHMR relation, this gives us
insight into the SF. In the next section we discuss the implications of 
these constraints in more detail. 

%%%%%%%%%%%%%%%%%%%%%%%%%%%%%%%%%%%%%
% Figure 5 
%%%%%%%%%%%%%%%%%%%%%%%%%%%%%%%%%%%%%
\begin{figure}
\begin{center}
\includegraphics[width=0.5\textwidth]{figs/sigMstar_tduty.pdf}
\caption{}
\label{fig:sigMstar_duty}
\end{center}
\end{figure}
%%%%%%%%%%%%%%%%%%%%%%%%%%%%%%%%%%%%%

\section{Results}
\subsection{The duty cycle of star formation}
In the previous section we describe our model constructed from blah with different
prescriptions for the star formation histories of star-forming galaxies. As the 
right panel of Figure~\ref{fig:abc_demo} demonstrates, different prescriptions for 
the star-forming galaxy star formation predicts different $\sigma_{\log\,M_*}$ 

\bitem
\item We find that by decreasing the timescale of stochasticity on a simple SFH model that traces the overall 
SFMS evolution does in fact decrease the scatter seen in the SMHMR. However, even with timescales less
than XXXX, we cannot reproduce observations. Ultimately to reproduce observations, we need to add in 
assembly bias. 

\eitem

Figure~\ref{fig:sigMstar_duty} 

%%%%%%%%%%%%%%%%%%%%%%%%%%%%%%%%%%%%%
% Figure 6 
%%%%%%%%%%%%%%%%%%%%%%%%%%%%%%%%%%%%%
\begin{figure}
\begin{center}
\includegraphics[width=0.5\textwidth]{figs/sigMstar_tduty.pdf}
\caption{}
\label{fig:abc_demo}
\end{center}
\end{figure}
%%%%%%%%%%%%%%%%%%%%%%%%%%%%%%%%%%%%%

\subsection{The need for a galaxy assembly bias}
\bitem
\item discuss how $t_{duty}$ is not enough to be consistent with $\sigma_{M_*}$. 
\item first clarify what you mean by galaxy assembly bias 
\item discuss implementation of galaxy assembly bias
\item Figure (pedagogical) of dlogSFR versus dMh dt for different correlation amounts 
\item Figure of different tdelay and dtabias 
\item Figure of sigma M star as a function of duty cycle and realistic dt abias and t delay 
\eitem

\section{Discussion} \label{sec:discussion}
\subsection{Rethinking the Main Sequence?}
\bitem 
\item Test the SMHMR for Louis's SFHs 
\eitem 

\section{Summary} \label{sec:summary}


\appendix
\section{$z \sim 1$ observations} \label{app:z1}
Much of the results presented in this paper are based on comparison 
between our model and observations at $z \sim 0.$. Our model is initalized 
at $z \sim 1$. Therefore, in this section we test some of the choices 
we make in our intializations. 

\bitem
\item Test impact of $z \sim 1$ SMF
\item Test impact of $z \sim 1$ $\sigma_{\log M_*}$ 
\eitem

%%%%%%%%%%%%%%%%%%%%%%%%%%%%%%%%%%%%%%%%%%%%%%%%%%%%%%%%%%%%%%%
% Acknowledgements
%%%%%%%%%%%%%%%%%%%%%%%%%%%%%%%%%%%%%%%%%%%%%%%%%%%%%%%%%%%%%%%
\section*{Acknowledgements}
It's a pleasure to thank 
    Louis Abramson, 
    \todo{more acknowledgements} 
for valueable discussions. 

\bibliographystyle{yahapj}
\bibliography{centralMS}
\end{document}
