\documentclass[iop,apj,tighten,twocolappendix,numberedappendix]{emulateapj}
\usepackage[breaklinks,colorlinks, urlcolor=blue,citecolor=blue,linkcolor=blue]{hyperref}
%define general packages
\usepackage{color}
\usepackage{epsfig}
\usepackage{rotating}
\usepackage{amsmath}
\usepackage{footnote}
\usepackage{setspace}
\usepackage{courier}
\usepackage{tabularx}
\usepackage{ctable, dashrule}

\newcommand\tab[1][1cm]{\hspace*{#1}}
\newcommand{\todo}[1]{{\bf \textcolor{red}{ #1}}}
\newcommand{\edited}[1]{\textcolor{red}{ #1}}
\definecolor{amethyst}{rgb}{0.6, 0.4, 0.8}
\definecolor{dred}{rgb}{0.75, 0., 0.0}
\definecolor{darkgreen}{rgb}{0., 0.75, 0.0}
\newcommand{\tocite}[1]{{\em \textcolor{amethyst}{ #1}}}
\newcommand{\beq}{\begin{equation}}
\newcommand{\eeq}{\end{equation}}
\newcommand{\overbar}[1]{\mkern 1.5mu\overline{\mkern-1.5mu#1\mkern-1.5mu}\mkern 1.5mu}
\newcommand{\avgSFR}{\overline{\raisebox{0pt}[1.2\height]{SFR}}}
\newcommand{\SFR}{\mathrm{SFR}}
\newcommand{\fq}{f_\mathrm{Q}}
\newcommand{\fqcen}{f_\mathrm{Q}^\mathrm{cen}}
\newcommand{\zinit}{z_\mathrm{initial}}
\newcommand{\taucen}{\tau_\mathrm{Q}^\mathrm{cen}}

\begin{document}

\title{} 

\author{ChangHoon~Hahn\altaffilmark{1}, 
Jeremy L.~Tinker\altaffilmark{2}, 
Andrew R.~Wetzel\altaffilmark{3,4,5}}
\altaffiltext{1}{Lawrence Berkeley National Laboratory, One Cyclotron Road, Berkeley, CA 94720, USA; changhoonhahn@lbl.gov}
\altaffiltext{2}{Center for Cosmology and Particle Physics, Department of Physics, New York University, 4 Washington Place, New York, NY 10003}
\altaffiltext{3}{TAPIR, California Institute of Technology, Pasadena, CA USA}
\altaffiltext{4}{Carnegie Observatories, Pasadena, CA USA}
\altaffiltext{5}{Department of Physics, University of California, Davis, CA USA}
\begin{abstract}
    Insert abstract Here 
\end{abstract}
\keywords{methods: numerical -- galaxies: clusters: general -- 
galaxies: groups: general -- galaxies: evolution -- galaxies: haloes -- 
galaxies: star formation -- cosmology: observations.}

\section{Introduction}
\begin{figure}
\begin{center}
\includegraphics[width=0.45\textwidth]{figs/t_quenching_comparison_z0_2.pdf}
\caption{from morphological quenching is in good agreement with our timescale.}
\label{fig:tquench_comp}
\end{center}
\end{figure}

\subsection{Quenching Star Formation in Central Galaxies}  

\section{Summary} \label{sec:summary}

%%%%%%%%%%%%%%%%%%%%%%%%%%%%%%%%%%%%%%%%%%%%%%%%%%%%%%%%%%%%%%%
% Acknowledgements
%%%%%%%%%%%%%%%%%%%%%%%%%%%%%%%%%%%%%%%%%%%%%%%%%%%%%%%%%%%%%%%
\section*{Acknowledgements}

\bibliographystyle{yahapj}
\bibliography{cenque}
\end{document}
