\documentclass[12pt, letterpaper, preprint]{aastex}
\usepackage{hyperref}
%%% This file is generated by the Makefile.
\newcommand{\giturl}{\url{https://github.com/changhoonhahn/centralMS}}
\newcommand{\githash}{9119283}\newcommand{\gitdate}{2019-04-05}\newcommand{\gitauthor}{changhoonhahn}


% typesetting shih
\linespread{1.08} % close to 10/13 spacing
\setlength{\parindent}{1.08\baselineskip} % Bringhurst
\setlength{\parskip}{0ex}
\let\oldbibliography\thebibliography % killin' me.
\renewcommand{\thebibliography}[1]{%
  \oldbibliography{#1}%
  \setlength{\itemsep}{0pt}%
  \setlength{\parsep}{0pt}%
  \setlength{\parskip}{0pt}%
  \setlength{\bibsep}{0ex}
  \raggedright
}
\setlength{\footnotesep}{0ex} % seriously?

\newcommand\tab[1][1cm]{\hspace*{#1}}
\newcommand{\todo}[1]{{\bf \textcolor{red}{ #1}}}
\newcommand{\edited}[1]{\textcolor{red}{ #1}}
\newcommand{\tocite}[1]{{\em \textcolor{amethyst}{ #1}}}
\newcommand{\beq}{\begin{equation}}
\newcommand{\eeq}{\end{equation}}
\newcommand{\overbar}[1]{\mkern 1.5mu\overline{\mkern-1.5mu#1\mkern-1.5mu}\mkern 1.5mu}
\newcommand{\avgSFR}{\overline{\raisebox{0pt}[1.2\height]{SFR}}}
\newcommand{\SFR}{\mathrm{SFR}}
\newcommand{\fq}{f_\mathrm{Q}}
\newcommand{\fqcen}{f_\mathrm{Q}^\mathrm{cen}}
\newcommand{\zinit}{z_\mathrm{initial}}
\newcommand{\taucen}{\tau_\mathrm{Q}^\mathrm{cen}}
\newcommand{\bitem}{\begin{itemize}}
\newcommand{\eitem}{\end{itemize}}

\begin{document}\sloppy\sloppypar\frenchspacing

\title{Central Galaxies on the Main Sequence} 
\date{\texttt{DRAFT~---~\githash~---~\gitdate~---~NOT READY FOR DISTRIBUTION}}
\author{ChangHoon~Hahn\altaffilmark{1}, 
Jeremy L.~Tinker\altaffilmark{2}, 
Andrew R.~Wetzel\altaffilmark{3,4,5}}
\altaffiltext{1}{Lawrence Berkeley National Laboratory, 1 Cyclotron Road, Berkeley, CA 94720}
\altaffiltext{2}{Center for Cosmology and Particle Physics, Department of Physics, New York University, 4 Washington Place, New York, NY 10003}
\altaffiltext{3}{TAPIR, California Institute of Technology, Pasadena, CA USA}
\altaffiltext{4}{Carnegie Observatories, Pasadena, CA USA}
\altaffiltext{5}{Department of Physics, University of California, Davis, CA USA}
\email{changhoon.hahn@lbl.gov}

\begin{abstract}
    Insert abstract Here 
\end{abstract}
\keywords{methods: numerical -- galaxies: clusters: general -- 
galaxies: groups: general -- galaxies: evolution -- galaxies: haloes -- 
galaxies: star formation -- cosmology: observations.}


{\bf Checklist} 
\bitem
\item Check the correlation between halo growth rate with different $t_{delay}$ and $\delta t_{abias}$ with the total halo growth rate between $z \sim 0$ and $z \sim 1$. 
\eitem 

\section{Introduction}
\bitem 
\item Motivate why we think SF galaxies evolve along the main sequence  
\item Discuss the current thought process on galaxy assembly bias 
\item Explain the limitation of SFH derivable from observations (Claire's fisher matrix paper would be really good; ask her about the details) 
\item Observations also can't provide detail host dark matter halo properties
\item So the approach with combining observations with N-body (empirical modeling) is very effective in the context of the halo.
\item Maybe talk about how the bigger context of why this is important?  
\eitem 
%\begin{figure}
%\begin{center}
%\includegraphics[width=0.45\textwidth]{figs/t_quenching_comparison_z0_2.pdf}
%\caption{from morphological quenching is in good agreement with our timescale.}
%\label{fig:tquench_comp}
%\end{center}
%\end{figure}
%
\section{}  

\section{Star Forming Central Galaxies}  
\subsection{Selecting $z \sim 0$ Star Forming Central Galaxies}  
\begin{itemize}
    \item Describe how $f_{SFMS}$ is calculated. 
    \item refer to Tjitske in prep 
    \item Then explain how it's not circular because the integrated $M_*$ has to reproduce the same SMF
\end{itemize}

\subsection{Evolving along the Main Sequence} 
\begin{itemize}
    \item Talk about the SFR and $M_*$ prescriptions 
    \item Pedagogical figure describing the SFH model: two panel one SFR as a function of time, the other residual log SFR as a funciton of time for different prescriptions
\end{itemize}

\section{The duty cycle of star formation}
\bitem
\item Figure that illustrates the fit to observables 
\item Figure of sigma M star as a function of duty cycle compared to observations 
\eitem 

\section{The need for a galaxy assembly bias}
\bitem
\item discuss how $t_{duty}$ is not enough to be consistent with $\sigma_{M_*}$. 
\item first clarify what you mean by galaxy assembly bias 
\item discuss implementation of galaxy assembly bias
\item Figure (pedagogical) of dlogSFR versus dMh dt for different correlation amounts 
\item Figure of different tdelay and dtabias 
\item Figure of sigma M star as a function of duty cycle and realistic dt abias and t delay 
\eitem

\section{Rethinking the Main Sequence?}
\bitem 
\item Test the SMHMR for Louis's SFHs 
\eitem 

\section{Summary} \label{sec:summary}

%%%%%%%%%%%%%%%%%%%%%%%%%%%%%%%%%%%%%%%%%%%%%%%%%%%%%%%%%%%%%%%
% Acknowledgements
%%%%%%%%%%%%%%%%%%%%%%%%%%%%%%%%%%%%%%%%%%%%%%%%%%%%%%%%%%%%%%%
\section*{Acknowledgements}

\bibliographystyle{yahapj}
\bibliography{cenque}
\end{document}
