\documentclass[12pt, letterpaper, preprint]{aastex}
\usepackage[breaklinks,colorlinks, urlcolor=blue,citecolor=blue,linkcolor=blue]{hyperref}
\usepackage{hyperref}
\usepackage{color}
%%% This file is generated by the Makefile.
\newcommand{\giturl}{\url{https://github.com/changhoonhahn/centralMS}}
\newcommand{\githash}{d9ea498}\newcommand{\gitdate}{2019-05-13}\newcommand{\gitauthor}{changhoonhahn}


% typesetting shih
\linespread{1.08} % close to 10/13 spacing
\setlength{\parindent}{1.08\baselineskip} % Bringhurst
\setlength{\parskip}{0ex}
\let\oldbibliography\thebibliography % killin' me.
\renewcommand{\thebibliography}[1]{%
  \oldbibliography{#1}%
  \setlength{\itemsep}{0pt}%
  \setlength{\parsep}{0pt}%
  \setlength{\parskip}{0pt}%
  \setlength{\bibsep}{0ex}
  \raggedright
}
\setlength{\footnotesep}{0ex} % seriously?

\newcommand\tab[1][1cm]{\hspace*{#1}}
\newcommand{\todo}[1]{{\bf \textcolor{red}{#1}}}
\newcommand{\beq}{\begin{equation}}
\newcommand{\eeq}{\end{equation}}
\newcommand{\overbar}[1]{\mkern 1.5mu\overline{\mkern-1.5mu#1\mkern-1.5mu}\mkern 1.5mu}
\newcommand{\avgSFR}{\overline{\raisebox{0pt}[1.2\height]{SFR}}}
\newcommand{\SFR}{\mathrm{SFR}}
\newcommand{\fq}{f_\mathrm{Q}}
\newcommand{\fqcen}{f_\mathrm{Q}^\mathrm{cen}}
\newcommand{\zinit}{z_\mathrm{initial}}
\newcommand{\taucen}{\tau_\mathrm{Q}^\mathrm{cen}}
\newcommand{\bitem}{\begin{itemize}}
\newcommand{\eitem}{\end{itemize}}

\begin{document}\sloppy\sloppypar\frenchspacing

%\title{Central Galaxies on the Main Sequence} 
\title{Star Formation Main Sequence in a Hierarchical Universe} 
\date{\texttt{DRAFT~---~\githash~---~\gitdate~---~NOT READY FOR DISTRIBUTION}}
\author{ChangHoon~Hahn\altaffilmark{1}, 
Jeremy L.~Tinker\altaffilmark{2}, 
Andrew R.~Wetzel\altaffilmark{3,4,5}}
\altaffiltext{1}{Lawrence Berkeley National Laboratory, 1 Cyclotron Road, Berkeley, CA 94720}
\altaffiltext{2}{Center for Cosmology and Particle Physics, Department of Physics, New York University, 4 Washington Place, New York, NY 10003}
\altaffiltext{3}{TAPIR, California Institute of Technology, Pasadena, CA USA}
\altaffiltext{4}{Carnegie Observatories, Pasadena, CA USA}
\altaffiltext{5}{Department of Physics, University of California, Davis, CA USA}
\email{changhoon.hahn@lbl.gov}

\begin{abstract}
    \todo{motivation, methodology, impact.}
    In observations star forming galaxies form a tight $log\;M_*$ to $log\;SFR$ 
    relation referred to as the {\em star formation main sequence} (SFMS) out to $z\sim2$. 
    Beyond the evolution ``along'' this SFMS, however, the star formation histories of star 
    forming galaxies have not been precisely characterized. 
    The SFH of these galaxies govern SMF, SFMS, and also observed constraints on the stellar mass to halo mass
    relation. 

    By combining high-resolution cosmological $N$-body simulation with observed evolutionary 
    trends of SF galaxies, we construct a model that tracks the evolution of star forming 
    central galaxies over the redshift $z < 1$. Comparing this model 

    Observations find a remarkably small scatter in the stellar mass to halo mass relation. 
    Somehow the star formation histories of galaxies must 
    
    According to observations, star forming galaxies form a tight $log\;M_*$ to $log\;SFR$ 
    relation referred to as the ``star formation main sequence'' out to $z\sim2$. 
\end{abstract}
\keywords{methods: numerical -- galaxies: clusters: general -- 
galaxies: groups: general -- galaxies: evolution -- galaxies: haloes -- 
galaxies: star formation -- cosmology: observations.}

We find that by decreasing the timescale of stochasticity on a simple SFH model that traces the overall 
SFMS evolution does in fact decrease the scatter seen in the SMHMR. However, even with timescales less
than XXXX, we cannot reproduce observations. Ultimately to reproduce observations, we need to add in 
assembly bias. 

{\bf Checklist} 
\bitem
\item Check the correlation between halo growth rate with different $t_{delay}$ and $\delta t_{abias}$ with the total halo growth rate between $z \sim 0$ and $z \sim 1$. 
\eitem 

\section{Introduction}
\bitem 
\item Motivate why we think SF galaxies evolve along the main sequence  
\item Discuss the current thought process on galaxy assembly bias 
\item Explain the limitation of SFH derivable from observations (Claire's fisher matrix paper would be really good; ask her about the details) 
\item Observations also can't provide detail host dark matter halo properties
\item So the approach with combining observations with N-body (empirical modeling) is very effective in the context of the halo.
\item Maybe talk about how the bigger context of why this is important?  
\item Why only centrals -- because our current best understanding of satellites is that they quench after infall, so it doesn't make sense to look at them
\item our model goes from $z < 1$ because beyond that the observations are statistically meaningless.  
\eitem 

%%%%%%%%%%%%%%%%%%%%%%%%%%%%%%%%%%%%%
% Figure 1 
%%%%%%%%%%%%%%%%%%%%%%%%%%%%%%%%%%%%%
\begin{figure}
\begin{center}
\includegraphics[width=0.9\textwidth]{figs/groupcat.pdf}
    \caption{Central galaxies in the SDSS DR7 group catalog. \emph{Left:} . 
    \emph{Right:}
    Fitting of the SFMS.}
\label{fig:groupcat}
\end{center}
\end{figure}
%%%%%%%%%%%%%%%%%%%%%%%%%%%%%%%%%%%%%

\section{Central Galaxies of SDSS DR7} \label{sec:sdss}
We construct our galaxy sample following the sample selection of \cite{tinker2011}. 
We select a volume-limited sample of galaxies with $M_r −5 log(h) < −18$ and complete in
$M_* > 10^{9.4} M_\odot$ from the NYU Value-Added Galaxy Catalog \citep[VAGC;][]{blanton2005}
of the Sloan Digital Sky Survey Data Release 7~\citep[SDSS DR7;][]{abazajian2009} at 
$z \approx 0.04$. The stellar masses of these galaxies are estimated using the
$\mathtt{kcorrect}$ code~\citep{blanton2007} assuming a~\cite{chabrier2003} initial
mass function. The star formation of the galaxies are estimated spectroscopically using the
specific star formation rates (SSFR) from the current release of the MPA-JHU spectral 
reductions\footnote{http://wwwmpa.mpa-garching.mpg.de/SDSS/DR7/}~\citep{brinchmann2004}.
Generally speaking, $\mathrm{SSFR} > 10^{-11}\mathrm{yr}^{-1}$ are derived from 
$\mathrm{H}_\alpha$ emission, $10^{-11} > \mathrm{SSFR} > 10^{-12}\mathrm{yr}^{-1}$
are derived from a combination of emission lines, and $\mathrm{SSFR} < 10^{-12}\mathrm{yr}^{-1}$
are based on $D_n 4000$~\citep[see discussion in][]{wetzel2013}. We note that 
$\mathrm{SSFR} < 10^{-12}\mathrm{yr}^{-1}$ should only be considered upper limits 
to the true galaxy SSFR~\citep{salim2007}.

From our galaxy sample, we identify the central galaxies using the \cite{tinker2011} halo-based 
group-finding algorithm, which is based on the~\cite{yang2005} algorithm and tested 
in~\cite{campbell2015}. The algorithm assigns a probability of being a satellite,
$P_\mathrm{sat}$, to each galaxy in the sample. Galaxies with $P_\mathrm{sat} \geq 0.5$ 
are classified as satellites and $P_\mathrm{sat} < 0.5$ are classified as centrals. 
In this paper we focus on central galaxies. With any group finding algorithm, galaxies are 
misassigned due to projection effects and redshift space distortions. The purity 
of the full central galaxy sample is $\sim 90\%$ with a completeness of $\sim 95\%$~\citep{tinker2017}.
Furthermore, \cite{campbell2015} find that the algorithm robustly identifies red and blue centrals
as a function of stellar mass, which is highly relevant to our analysis.  

In the left panel of Figure~\ref{fig:groupcat}, we plot the SFR-$M_*$ distribution of
the SDSS DR7 central galaxies. In the right panel, we plot the distribution of SSFR, 
$p(\log \mathrm{SSFR})$, for galaxies with $10.6 < \log \,M_* < 10.8$ (stellar mass range 
highlighted on the left panel). Both panels of Figure~\ref{fig:groupcat} illustrate the 
bimodality in the galaxy sample. The SFR-$M_*$ distribution also illustrate the correlation
between SFR and $M_*$ in star-forming galaxies \emph{i.e.} the star-formation main sequence 
(SFMS).

\section{Model: Simulated Central Galaxies} \label{sec:sim}
We're interesting in constructing a model that tracks central galaxies and 
their star formation within the heirarchical growth of their host halos. This 
requires a cosmological $N$-body simulation that accounts for the complex 
dynamical processes that govern the host halos of galaxies. In this paper 
we use the high resolution $N$-body simulation from~\cite{wetzel2013} generated 
using the \cite{white2002} $\mathtt{TreePM}$ code with flat $\Lambda$CDM cosmology 
($\Omega_m =0.274, \Omega_b = 0.0457, h = 0.7, n=0.95, \mathrm{and} \sigma_8 = 0.8$).
From initial conditions at $z = 150$ generated from second-order Lagrangian 
Perturbation Theory, $2048^3$ particles with mass of $1.98 \times 10^8\,M_\odot$ are 
evolved in a $250 \mathrm{Mpc}/h$ box with a Plummer equivalent smoothing of 
$2.5\,\mathrm{kpc}/h$. For a more detailed description of the simulation, we 
refer readers to~\cite{wetzel2013, wetzel2014}.

From the $\mathrm{TreePM}$ $N$-body simulation, `host halos' are identified 
using the Friends-of-Friends (FoF) algorithm of \cite{davis1985} with 
linking length of $b = 0.168$ times the mean inter-partcile spacing. Within 
these host halos, \cite{wetzel2013} identifies `subhalos' as overdensities 
in phase space through a six-dimensional FoF algorithm~\citep[FoF6D][]{white2010}. 
The host halos and subhalos are then tracked across the $45$ simulation 
outputs from $z = 10$ to $0$ to build merger trees~\citep{wetzel2009,wetzel2010}. 
The most massive subhalos in newly-formed host halos at a given simulation 
output are defined as the `central' subhalo. A central subhalo retains its 
`central' definition until it falls into a more massive host halo, at which 
point it becomes a `satellite' subhalo. 

Each subhalo is assigned a $M_\mathrm{peak}$, the maxmum host halo mass that 
it ever had as a central subhalo. Using $M_\mathrm{peak}$, we construct a galaxy 
catalog from the subhalos using subhalo abundance 
matching~\citep[SHAM;][]{conroy2006,vale2006,yang2009,wetzel2012,leja2013,wetzel2013,wetzel2014,hahn2017a}. 
In principle, SHAM assumes a one-to-one mapping between subhalo 
$M_\mathrm{peak}$ and galaxy stellar mass $M_*$: $n(> M_\mathrm{peak}) > n(> M_*)$
that preserves the rank ordering. In practice, we apply a $0.2$ dex log-normal 
scatter in $M_∗$ at fixed $M_\mathrm{peak}$ based on observations of the stellar 
mass to halo mass relation (SMHMR; \todo{bunch of SMHMR citations}). \cite{gu2016} 
compile empirical constraints on the scatter of this stellar mass to halo 
mass relation ($\sigma_{\log M_*}$). Using the SHAM mapping, we can 
assign galaxy stellar mass to subhalos based on observed stellar mass 
functions (SMFs) at the redshifts of the simulation outputs (snapshots). 

We use the SMF from \cite{li2009} at $z = 0.05$ and at higher redshifts 
interpolate between the \cite{li2009} SMF and the SMF from \cite{marchesini2009} 
at $z = 1.6$. We choose the \cite{li2009} SMF because it is based on the 
same SDSS NYU-VAGC sample as our SDSS DR7 group catalog (Section~\ref{sec:sdss}). 
We choose the \cite{marchesini2009} SMF, amongst others, because it produces 
interpolated SMFs that monotonically increase at $z < 1$. As noted in 
\cite{hahn2017a}, at $z \approx 1$, the SMF interpolated between the 
\cite{li2009} and \cite{marchesini2009} SMFs is consistent with more 
recent measurements from \cite{muzzin2013} and \cite{ilbert2013}. 
At each snapshot, we independently use SHAM to assign galaxy $M_*$. 
This way, we not only track the evolution of subhalos, but also the 
the galaxies' $M_*$. With the $45$ snapshots outputs from our simulation, 
we can in principle track the central galaxies back to $z \sim 10$. 
However, we restrict ourselves to snapshots at $z \lesssim 1$, where we 
have the most statistically meaningful observations. We next describe
how we select star forming central galaxies in our model and initalize them. 
%As we desribe later in this section, however, we're only interested in the snapshots at $z \approx 0.05$ and $1$, respectively. 
\todo{TBD: Perhaps mention in appendix how we test different SMF assumptions}% (In Section~\ref{app:z1},)

%%%%%%%%%%%%%%%%%%%%%%%%%%%%%%%%%%%%%
% Figure 2 
%%%%%%%%%%%%%%%%%%%%%%%%%%%%%%%%%%%%%
%\begin{figure}
%\begin{center}
%\includegraphics[width=0.5\textwidth]{figs/fq_fsfms.pdf}
%\caption{SFMS fraction versus quiescent fraction from Hahn}
%\label{fig:fq_fsfms}
%\end{center}
%\end{figure}
%%%%%%%%%%%%%%%%%%%%%%%%%%%%%%%%%%%%%

\subsection{Selecting Star Forming Central Galaxies}  
In our model, we're interested in tracking the evolution of the
SFR and stellar mass of SF central galaxies. To construct such a model, 
we first need to select star-forming galaxies from the central galaxies 
in our simulation described earlier in this section. Since we later 
compare our model to observation, our selection is based on 
$f^\mathrm{cen}_\mathrm{SFMS}(M_*)$, the fraction of central galaxies 
within the star-forming main sequence, measured from the SDSS DR7 VAGC 
(Section~\ref{sec:sdss}). Below, we describe how we derive this 
$f^\mathrm{cen}_\mathrm{SFMS}(M_*)$ and use it to select star-forming 
central galaxies in our model. Afterwards we describe how we initalize 
the SFRs and $M_*$ of these galaxies in our model.

Often in the literature, an empirical color-color or SFR-$M_*$ cut 
that separates the two main modes (red/blue or star-forming/quiescent) 
in the distribution is chosen to classify 
galaxies~\citep[\emph{e.g.}][]{baldry2006, blanton2009, drory2009, peng2010, moustakas2013, hahn2015}.
The red/quiescent or blue/star-forming fractions derived from this sort of 
classification, by construction, depend on the choice of cut and neglect the 
transitioning galaxies~\emph{i.e.} the galaxies in the ``green valley''. 
For our $f^\mathrm{cen}_\mathrm{SFMS}(M_*)$ measurement, we instead use a
method frm Tjitske et al.~(in prep), which is based on fitting the 
SFMS from the SFR-$M_*$ distribution 

The SFMS fitting scheme first divides the SDSS DR7 VAGC central 
galaxy sample (Section~\ref{sec:sdss}) into stellar mass bins of 
width $\Delta \log M_* = 0.2~\mathrm{dex}$. We then fit the SSFR 
distribution of each bin using Gaussian mixture models (GMMs) with 
$1 - 3$ components using the expectation-maximization 
algorithm~\citep[EM;][]{dempster1977, neal1998}. We restrict 
ourselves to models with a maximum of $3$ components for the three 
possible galaxy classifications: quiescent, star-forming, and green 
valley populations. From the three GMMs, we select the model with 
the lowest Bayesian Information Criteria~\citep[BIC][]{schwarz1978}. 
The Gaussian component of this GMM with mean $\log \mathrm{SSFR} > -11$ 
is identified as the SFMS. In the rare cases when more than one GMM 
component has mean $\log \mathrm{SSFR} > -11$, we compare the weights 
of the components.  If the weight of one component is less than a 
third of the other, we take the component with the higher weight to 
represent the SFMS. Otherwise, we omit the stellar mass bin altogether. 
The weight of the SFMS GMM component provides an estimate of 
$f^\mathrm{cen}_\mathrm{SFMS}$ for the given stellar mass bin. 
In the right panel of Figure~\ref{fig:groupcat}, we plot the SFMS 
GMM component (blue shaded region) of the $p(\log \mathrm{SSFR})$ 
for the SDSS DR7 central galaxies within $10.6 < \log M_* < 10.8$. 
The SFMS constitutes $f^\mathrm{cen}_\mathrm{SFMS} = 0.21$ of the 
SDSS central galaxies in this stellar mass bin. 

Next, with the $f^\mathrm{cen}_\mathrm{SFMS}$ values spanning 
the different stellar mass bins, we fit $f^\mathrm{cen}_\mathrm{SFMS}$ as 
a function of $\log\,M_*$. Using a fiducial stellar mass of 
$\log\,M_\mathrm{fid} = 10.5$, we derive the following best-fit 
\beq \label{eq:f_cen_sfms}
f^\mathrm{cen}_\mathrm{SFMS, bestfit}(M_*) = -0.627\,(\log\,M_* - 10.5) + 0.354. 
\eeq
We note that this $f^\mathrm{cen}_\mathrm{SFMS, bestfit}(M_*)$ is 
in good agreement with the $f_\mathrm{Q}^\mathrm{cen}(M_*; z\sim0)$ 
fit from \cite{hahn2017a}. For each central galaxy in our simulation, 
we assign a probability of it being on the SFMS, using 
$f^\mathrm{cen}_\mathrm{SFMS, bestfit}(M_*)$ with $M_*$ at $z \sim 0$ 
assigned through SHAM. Based on these probablities, we randomly 
identify central galaxies from our simulation as star-forming at 
$z \sim 0$.

At this point we make the assumption that once a SF galaxy quenches 
its star formation, it remains quiescent. This means that all of the 
central galaxies that we place on the SFMS at $z \sim 0$, were also 
on the SFMS at $z \sim 1$ --- the initial redshift of our model. From 
SHAM we have the initial $M_*$s of our SF centrals at $z \sim 1$. We 
also need initial SFRs of these galaxies, which we can get from the 
SFR of the SFMS. The measurements of the SFMS in the literature, however,
have significant discrepancies. In \cite{speagle2014}, they compile 
SFRs of the SFMS from 25 studies in the literature, each with different 
methods of deriving galaxy SFR and $M_*$. Even after calibrating, the 
SFRs of the SFMS from different analyses at $z \sim 1$ for $M_* = 10^{10.5}\, M_\odot$ 
vary by more than a factor of 2~\citep[see Figure 2 of][]{speagle2014}. 
This is particularly concerning since the amplitude and slope of the SFMS 
significantly dictate the SFR and $M_*$ evolution of SF centrals in our 
model. In order to deal with this uncertainty, we use the following 
prescription for the SFMS SFR: $\log\,\mathrm{SFR}_\mathrm{MS}(M_*, z)$. 

At $z \sim 0$, we fix $\log\,\mathrm{SFR}_\mathrm{MS}(M_*, z=0)$ to the 
$\log\, \mathrm{SFR}(M_*)$ relation derived from fitting the SFMS 
of the SDSS DR 7 centrals. \todo{details} 
For $z > 0$, we include two free parameters to dictate the redshift
evolution of the amplitude and slope of the SFMS SFR-$M_*$ relation: 
\todo{param}, respectively. 
\beq
\log\,\mathrm{SFR}_\mathrm{MS}(M_*, z) = 
\eeq
Later in our analysis, we set the following
priors for \todo{params}: \todo{prior}. These priors are conservatively 
chosen to encompass the best-fit SFMS  from \cite{speagle2014} and also 
\todo{others}. 

%In order to initialize the SFRs of SF centrals at z~1 we refer to observations from XXXXX. These observations however derived their galaxy properties from various different methods than the VAGC.  Instead of dealing with the differences in observations, we include two free parameters to dictate the redshift evolution of the amplitude and slope of the SFMS. 



%\bitem
%\item Then explain how it's not circular because the integrated $M_*$ has to reproduce the same SMF?
%\eitem

\subsection{Evolving along the Main Sequence} 
\bitem
\item talk about how we assign the initial SFR using the SHAM M*
\item Talk about the SFR and $M_*$ prescriptions 
\item parameterization of SFMS 
\item explicitly talk about the free parameters of the model. 
\item priors for $\beta_M$ and $\beta_z$ encompass the observable constraints 
\item talk about inference using ABC
\eitem

\section{Results}
\subsection{The duty cycle of star formation}
\bitem
\item Figure that illustrates the fit to observables 
\item Figure of sigma M star as a function of duty cycle compared to observations 
\eitem 

\subsection{The need for a galaxy assembly bias}
\bitem
\item discuss how $t_{duty}$ is not enough to be consistent with $\sigma_{M_*}$. 
\item first clarify what you mean by galaxy assembly bias 
\item discuss implementation of galaxy assembly bias
\item Figure (pedagogical) of dlogSFR versus dMh dt for different correlation amounts 
\item Figure of different tdelay and dtabias 
\item Figure of sigma M star as a function of duty cycle and realistic dt abias and t delay 
\eitem

\section{Discussion} \label{sec:discussion}
\subsection{Rethinking the Main Sequence?}
\bitem 
\item Test the SMHMR for Louis's SFHs 
\eitem 

\section{Summary} \label{sec:summary}


\appendix
\section{$z \sim 1$ observations} \label{app:z1}
Much of the results presented in this paper are based on comparison 
between our model and observations at $z \sim 0.$. Our model is initalized 
at $z \sim 1$. Therefore, in this section we test some of the choices 
we make in our intializations. 

\bitem
\item Test impact of $z \sim 1$ SMF
\item Test impact of $z \sim 1$ $\sigma_{\log M_*}$ 
\eitem

\begin{figure}
\begin{center}
\includegraphics[width=0.9\textwidth]{figs/sfh_pedagogical.pdf}
\caption{Pedagogical figure that illustrates how star forming central galaxies in our model
evolve along the SFMS.}
\label{fig:sfh_model}
\end{center}
\end{figure}

\begin{figure}
\begin{center}
\includegraphics[width=0.9\textwidth]{figs/qaplot_abc.pdf}
%qaplot_abc_test0_t14.pdfqaplot_abc_randomSFH_short_t9.pdf}
\caption{}
\label{fig:abc_demo}
\end{center}
\end{figure}

\begin{figure}
\begin{center}
\includegraphics[width=0.5\textwidth]{figs/sigMstar_tduty.pdf}
\caption{}
\label{fig:abc_demo}
\end{center}
\end{figure}

%%%%%%%%%%%%%%%%%%%%%%%%%%%%%%%%%%%%%%%%%%%%%%%%%%%%%%%%%%%%%%%
% Acknowledgements
%%%%%%%%%%%%%%%%%%%%%%%%%%%%%%%%%%%%%%%%%%%%%%%%%%%%%%%%%%%%%%%
\section*{Acknowledgements}
Louis Abramson

\bibliographystyle{yahapj}
\bibliography{centralMS}
\end{document}
