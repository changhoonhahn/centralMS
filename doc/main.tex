\documentclass[12pt, letterpaper, preprint, tighten]{aastex62}
%\usepackage[breaklinks,colorlinks, urlcolor=blue,citecolor=blue,linkcolor=blue]{hyperref}
%\usepackage{hyperref}
\usepackage{amsmath}
\usepackage{color}
\usepackage{comment}

% typesetting shih
\linespread{1.08} % close to 10/13 spacing
\setlength{\parindent}{1.08\baselineskip} % Bringhurst
\setlength{\parskip}{0ex}
\let\oldbibliography\thebibliography % killin' me.
\renewcommand{\thebibliography}[1]{%
  \oldbibliography{#1}%
  \setlength{\itemsep}{0pt}%
  \setlength{\parsep}{0pt}%
  \setlength{\parskip}{0pt}%
  \setlength{\bibsep}{0ex}
  \raggedright
}
\setlength{\footnotesep}{0ex} % seriously?

\definecolor{orange}{rgb}{1,0.5,0}

\newcommand\tab[1][1cm]{\hspace*{#1}}
\newcommand{\todo}[1]{{\bf \textcolor{red}{#1}}}
\newcommand{\ch}[1]{\color{orange}{\bf CH:} #1}
\newcommand{\beq}{\begin{equation}}
\newcommand{\eeq}{\end{equation}}
\newcommand{\overbar}[1]{\mkern 1.5mu\overline{\mkern-1.5mu#1\mkern-1.5mu}\mkern 1.5mu}
\newcommand{\avgSFR}{\overline{\raisebox{0pt}[1.2\height]{SFR}}}
\newcommand{\SFR}{\mathrm{SFR}}
\newcommand{\fq}{f_\mathrm{Q}}
\newcommand{\fqcen}{f_\mathrm{Q}^\mathrm{cen}}
\newcommand{\zinit}{z_\mathrm{initial}}
\newcommand{\taucen}{\tau_\mathrm{Q}^\mathrm{cen}}
\newcommand{\logsfr}{\log\mathrm{SFR}}
\newcommand{\logsfrsfs}{\log\mathrm{SFR}_\mathrm{SFS}}
\newcommand{\bitem}{\begin{itemize}}
\newcommand{\eitem}{\end{itemize}}
\newcommand{\musfms}{\log\overline{\mathrm{SFR}}_\mathrm{SFS}}
\newcommand{\siglogm}{\sigma_{\log M_*}} 

\begin{document}\sloppy\sloppypar\frenchspacing

\title{The Star-Forming Sequence in a Hierarchical Universe} 
%\title{Central Galaxies on the Main Sequence} 
%\date{\texttt{DRAFT~---~\githash~---~\gitdate~---~NOT READY FOR DISTRIBUTION}}
\author{ChangHoon Hahn}
\altaffiliation{hahn.changhoon@gmail.com}
\affil{Lawrence Berkeley National Laboratory, 1 Cyclotron Rd, Berkeley CA 94720, USA}
\affil{Berkeley Center for Cosmological Physics, University of California, Berkeley, CA 94720, USA}
\affil{Center for Cosmology and Particle Physics, Department of Physics, New York University, 4 Washington Place, New York, NY 10003}
\author{Jeremy L.~Tinker} 
\affil{Center for Cosmology and Particle Physics, Department of Physics, New York University, 4 Washington Place, New York, NY 10003}
\author{Andrew R.~Wetzel}
\affil{TAPIR, California Institute of Technology, Pasadena, CA USA}
\affil{Carnegie Observatories, Pasadena, CA USA}
\affil{Department of Physics, University of California, Davis, CA USA}

\begin{abstract}
    \todo{sentence on the questions of SF galaxies.}
    Star-forming galaxies are observed to have a tight relationship between their 
    star  formation rates and stellar masses. This so-called ``star-forming 
    sequence'' (SFS) characterizes both the star formation histories and stellar 
    mass growth of star-forming galaxies. Meanwhile, observed constraints on the 
    stellar-to-halo mass relation (SHMR) connect the stellar mass growth of galaxies 
    to their host halo halo accretion history. Combining these observed trends with 
    a high-resolution cosmological $N$-body simulation, we present a model that 
    tracks the star formation, stellar mass, and host halo accretion histories of 
    star-forming central galaxies over $z < 1$. By comparing this model to the 
    observed stellar mass function and SFS of central galaxies in the SDSS Data 
    Release 7, we find that star formation variability on timescales $\lesssim 0.5\,\mathrm{Gyr}$ 
    is \emph{necessary} to produce tighter scatter in the SHMR, $\sigma_{\log M_*}$, 
    at $M_h{=}10^{12}M_\odot$ comparable to observations. However, to conservatively 
    reproduce the observed $\sigma_{\log M_*}{\sim}0.2\,\mathrm{dex}$, the star 
    formation histories must also correlate strongly with halo accretion history 
    with a correlation coefficient $r > 0.5$ --- \emph{i.e.} exhibit strong assembly 
    bias. The timescale of star formation variability and the correlation between star
    formation and halo accretion history we infer, provide key constraints on 
    the star formation history and the evolution of star-forming central galaxies.
    %Observations also place constraints on the stellar-to-halo mass relation (SMHR), which suggest that  the evolution of star forming galaxies is also inextricably linked to the underlying halo. 
    %The SFS alone, however, does not dictate the evolution star forming galaxies. 
    %Based on observed constraints on the stellar-to-halo mass relation (SHMR), halo  accretion history also likely plays a role in the evolution  of star-forming galaxies. 
    %By combining a high-resolution cosmological $N$-body  simulation with observed evolutionary trends of the SFS, we present a model that  tracks the star formation, stellar mass, and host halo mass histories of  star-forming central galaxies over $z < 1$ and 
\end{abstract}
\keywords{methods: numerical -- galaxies: clusters: general -- 
galaxies: groups: general -- galaxies: evolution -- galaxies: haloes -- 
galaxies: star formation -- cosmology: observations.}

\section{Introduction}
Observations from surveys such as the Sloan Digital Sky 
Survey~\citep[SDSS;][]{york2000}, Cosmic Evolution
Survey~\citep[COSMOS;][]{scoville2007},
and the PRIsm MUlti-object Survey~\citep[PRIMUS;][]{coil2011, cool2013} 
have provided statistically meaningful samples of galaxies, 
which have been critical for establishing the global trends of 
galaxies in the local universe. Broadly speaking, galaxies 
fall into two categories: quiescent and star-forming galaxies. 
Quiescent galaxies have little to no star formation, are red 
in color, and have ellipitical morphologies. Meanwhile, 
star-forming galaxies have significant star formation, are blue 
in color, and have disk-like morphologies (\citealt{kauffmann2003, blanton2003, baldry2006, taylor2009, moustakas2013}; 
see~\citealt{blanton2009} and references therein). 

Star-forming galaxies, furthermore, are found on the so-called 
``star-forming sequence'' (hereafter SFS), a tight relationship 
between their star formation rates (SFR) and stellar 
masses~\citep[][see also Figure~\ref{fig:groupcat}]{noeske2007, daddi2007, salim2007, speagle2014, lee2015}.
This sequence, which is observed out to $z > 2$~\citep{wang2013, 
leja2015, schreiber2015}, plays a crucial role in characterizing 
galaxy evolution over the past ${\sim}10\,\mathrm{Gyr}$. 
The significant fraction of star-forming galaxies that quench their 
star formation and migrate off of the SFS reflects the significant 
growth in the fraction of quiescent galaxies~\citep{blanton2006, borch2006, bundy2006, moustakas2013}. 
Meanwhile, the decline of star formation in the entire SFS~\citep{lee2015, schreiber2015} 
over time reflects the decline in overall cosmic star formation~\citep{hopkins2006, behroozi2013a, madau2014}.
The SFS and its evolution also connects star formation history
and stellar mass growth of star-forming galaxies. 


{\ch some blanket statements about connection between galaxy-halo connection} 
For instance, the stellar-to-halo mass relation (SHMR) is constrained
from 
\todo{papers that look at SHMR or sigma of that from high redsihft }
~\citep{mandelbaum2006a, conroy2007, more2011, leauthaud2012, tinker2013, velander2014, han2015, zu2015, gu2016}
Constraints on the SHMR at $z\sim 1$ from observations find 
Remarkably, at $z\sim 0$, observational constraints find little 
evolution in the scatter in the SHMR. 
These observational constraints suggest that stellar mass growth 
of galaxies are closely linked to the hierarchical growth of their 
host dark matter halos. 

Despite our understanding of the evolving galaxy population in a
hierarchical unvierse, we face a number of challenges when it comes 
to understanding their detailed star formation histories (SFH). 
For instance, SFHs at lookback times longer than $200\,\mathrm{Myr}$ 
do not contribute to SFR indicators such as $H\alpha$ or $FUV$ fluxes.
Measuring SFHs from fitting photometry or spectroscopy typically 
assume a specific functional form of the SFH, such as exponentially 
declining or lognormal, that do not include short timescale 
variability~\citep[\emph{e.g.}][]{wilkinson2017, carnall2018}. 
Even methods that recover non-parametric SFHs from high signal-to-noise 
observations can only retrieve SFHs in coarse temporal resolutions~\citep[\emph{e.g.}][]{tojeiro2009, leja2018a}. 
While simulations provide another means for understanding SFHs, 
are also subject to their specific time and mass resolutions that 
suppress the variability of their star formation, especially in 
analytic models, semi-analytic models, and large-volume cosmological 
simulations~\citep{sparre2017a}. These challenges make difficult to constrain certain things like 
the timescale of star formation variability.

{\ch paragraph on assembly bias and how we've look at it w.r.t. 
quenching and stuff but it's still unclear at a sfh level}.

In this paper, we take advantage of this connection between SFH and 
the underlying dark matter to constrain the SFH of star-forming central 
galaxies. We focus on central galaxies since the SFHs of satellites 
are likely influenced by environmentally-driven external mechanisms 
such as ram pressure stripping (Gunn \& Gott 1972; Bekki 2009), 
strangulation (Larson et al. 1980; Balogh et al. 2000), or 
harassment (Moore et al. 1998). Moreover, centrals constitute the 
majority of massive galaxies ($M_*>10^{9.5}M_\sun$) at 
$z\sim0$~\citep{wetzel2013}. Using a similar approach as \cite{wetzel2013, hahn2017}, 
we present a model that combines a high resolution cosmological 
$N$-body simulation with observed evolutionary trends of the SFS. 
Our model statistically tracks the star formation, stellar mass, 
and host halo mass histories of star-forming central galaxies over 
$z\sim1$ to $0$. By comparing predictions of this empirical 
model to observations, we constrain the timescale of star formation 
variability and also the correlation between SFH and host halo history.  
\todo{sentence on implication}

In Section~\ref{sec:sdss} we describe the $z\approx0$ central galaxy 
sample that we construct from SDSS Data Release 7. Then in 
Section~\ref{sec:sim}, we describe the $N$-body simulation and how we
evolve the SFR and stellar masses of the star-forming central galaxies 
in our model. We compare predictions from our model to observations 
and present the resulting constraints in Section~\ref{sec:results}. 
Finally, we conclude and summarize the results in 
Section~\ref{sec:summary}.


%%%%%%%%%%%%%%%%%%%%%%%%%%%%%%%%%%%%%
% Figure 1 
%%%%%%%%%%%%%%%%%%%%%%%%%%%%%%%%%%%%%
\begin{figure}
\begin{center}
\includegraphics[width=0.75\textwidth]{figs/groupcat.pdf}
    \caption{The SFR--$M_*$ relation of the central galaxies in SDSS DR7 
    mark the bimodal distribution of the star-forming and quiescent 
    populations (left panel). \emph{Star-forming centrals, based on the correlation between their 
    SFR and $M_*$, lie on the so-called ``star-forming sequence''}.
    %The transitioning galaxies lie on the ``green'' valley between the star-fomring and quiescent modes. 
    On the right, we present the SSFR distribution, $p(\log\mathrm{SSFR})$, 
    of SDSS centrals with $10.6 < \log M_* < 10.8$. Based on the SFS component
    from the \cite{hahn2018a} GMM fit to the SFR--$M_*$ relation (shaded in blue), 
    galaxies in the SFS account for $f_\mathrm{SFS} = 0.21$ of the centrals 
    in the stellar mass bin.} \label{fig:groupcat}
\end{center}
\end{figure}
%%%%%%%%%%%%%%%%%%%%%%%%%%%%%%%%%%%%%

\section{Central Galaxies of SDSS DR7} \label{sec:sdss}
We construct our galaxy sample following the sample selection of \cite{tinker2011}. 
We select a volume-limited sample of galaxies at $z \approx 0.04$ with 
$M_r - 5 \log(h) < -18$ and complete above $M_* > 10^{9.4} M_\odot$ from 
the NYU Value-Added Galaxy Catalog \citep[VAGC;][]{blanton2005} of the 
Sloan Digital Sky Survey Data Release 7~\citep[SDSS DR7;][]{abazajian2009}. 
The stellar masses of these galaxies are estimated using the
$\mathtt{kcorrect}$ code~\citep{blanton2007} assuming a~\cite{chabrier2003} 
initial mass function. For their specific star formation rates (SSFR) we use 
measurements from the current release of the MPA-JHU spectral 
reductions\footnote{http://wwwmpa.mpa-garching.mpg.de/SDSS/DR7/}~\citep{brinchmann2004}.
Generally, $\mathrm{SSFR} > 10^{-11}\mathrm{yr}^{-1}$ are derived from 
$\mathrm{H}\alpha$ emission, $10^{-11} > \mathrm{SSFR} > 10^{-12}\mathrm{yr}^{-1}$
are derived from a combination of emission lines, and $\mathrm{SSFR} < 10^{-12}\mathrm{yr}^{-1}$
are based on $D_n 4000$~\citep[see discussion in][]{wetzel2013}. We emphasize that 
$\mathrm{SSFR} < 10^{-12}\mathrm{yr}^{-1}$ should only be considered upper limits 
to the actual galaxy SSFR~\citep{salim2007}.

From this galaxy sample, we identify central galaxies using the 
\cite{tinker2011} group finder, a halo-based algorithm that uses 
the abundance matching ansatz to iteratively assign halo masses to groups. 
It initially assigns a tentative halo mass to each galaxy by matching their 
abundances. Then starting with the most massive galaxy, nearby lower
mass galaxies are assigned a probability of being a satellite. Once all 
the galaxies are assigned to a group, the halo masses of the central galaxies 
are updated by abundance matching with the total stellar mass in the groups. 
This entire process is repeated until convergence. Every group contains one 
central galaxy, which by definition is the most massive, and a group can 
contain zero, one, or many satellites.
%\todo{The algorithm assigns a probability of being a satellite, $P_\mathrm{sat}$, to each galaxy in the sample. Galaxies with $P_\mathrm{sat} \geq 0.5$  are classified as satellites and $P_\mathrm{sat} < 0.5$ are classified as centrals.} In this paper we focus on central galaxies. 
As with any group finder, galaxies are misassigned due to projection 
effects and redshift space distortions. Our central galaxy sample has
a purity of ${\sim}90\%$ and completeness of ${\sim}95\%$~\citep{tinker2017}
Moreover, as illustrated in \cite{campbell2015}, the \cite{tinker2011} group
finder robustly identifies red and blue centrals as a function of stellar mass, 
which is highly relevant to our analysis.  
We present the SFR--$M_*$ relation of the SDSS DR7 central galaxies, described 
above, in the left panel of Figure~\ref{fig:groupcat}. The contours of the 
relation clearly illustrate the bimodality in the galaxy sample with the 
star-forming centrals lying on the so-call ``star-forming sequence'' (SFS). 
% In the left panel of Figure~\ref{fig:groupcat}, we plot the SFR-$M_*$ distribution of the SDSS DR7 central galaxies. In the right panel, we plot the distribution of SSFR,  $p(\log \mathrm{SSFR})$, for galaxies with $10.6 < \log \,M_* < 10.8$ (stellar mass range  highlighted on the left panel). Both panels of Figure~\ref{fig:groupcat} illustrate the  bimodality in the galaxy sample. The SFR-$M_*$ distribution also illustrate the correlation between SFR and $M_*$ in star-forming galaxies \emph{i.e.} the star-formation main sequence  (SFS).

\section{Model: Simulated Central Galaxies} \label{sec:sim}
We're interesting in constructing a model that tracks central galaxies and 
their star formation within the heirarchical growth of their host halos. This 
requires a cosmological $N$-body simulation that accounts for the complex 
dynamical processes that govern the host halos of galaxies. In this paper 
we use the high resolution $N$-body simulation from~\cite{wetzel2013} generated 
using the \cite{white2002} $\mathtt{TreePM}$ code with flat $\Lambda$CDM cosmology 
($\Omega_m =0.274, \Omega_b = 0.0457, h = 0.7, n=0.95$, and $\sigma_8 = 0.8$).
From initial conditions at $z = 150$, generated from second-order Lagrangian 
Perturbation Theory, $2048^3$ particles with mass of $1.98 \times 10^8\,M_\odot$ are 
evolved in a $250\,\mathrm{Mpc}/h$ box with a Plummer equivalent smoothing of 
$2.5\,\mathrm{kpc}/h$~\citep{wetzel2013, wetzel2014}. `Host halos' are then 
identified using the Friends-of-Friends algorithm~\citep[FoF;][]{davis1985} with 
linking length of $b{=}0.168$ times the mean inter-partcile spacing. Within 
these host halos, \cite{wetzel2013} identifies `subhalos' as overdensities 
in phase space through a six-dimensional FoF algorithm~\citep[FoF6D;][]{white2010}. 
The host halos and subhalos are then tracked across the simulation outputs 
from $z = 10$ to $0$ to build merger trees~\citep{wetzel2009,wetzel2010}. 
The most massive subhalos in newly-formed host halos at a given simulation 
output are defined as the `central' subhalo. A central subhalo retains its 
`central' definition until it falls into a more massive host halo, at which 
point it becomes a `satellite' subhalo. 

Throughout its $45$ snapshot outs, $\mathtt{TreePM}$ simulation tracks 
the evolution of subhalos back to $z \sim 10$. We restrict ourselves to $15$ 
snapshots from $z = 1.08$ to $0.05$, where we have the most statistically 
meaningful observations. Furthermore, since we're interested in centrals we only 
keep subhalos that are classified as centrals throughout the redshift 
range. This criterion removes ``black splash'' or ``ejected'' satellite 
galaxies~\citep[\emph{e.g.}][]{mamon2004,wetzel2014} misclassified as 
centrals. Next, we describe how we select and initialize the star-forming 
central galaxies from the central subhalos of the $\mathtt{TreePM}$ 
simulation in our model.

%%%%%%%%%%%%%%%%%%%%%%%%%%%%%%%%%%%%%%%%%%%%%%%%%%%%%%%%%%%%%%%%%%%%%%%%%%
% Section 
%%%%%%%%%%%%%%%%%%%%%%%%%%%%%%%%%%%%%%%%%%%%%%%%%%%%%%%%%%%%%%%%%%%%%%%%%%
\subsection{Selecting Star-Forming Centrals}  \label{sec:sfcen}
To construct a model that tracks the SFR and stellar mass evolution of 
star-forming central galaxies, we first need to select them from the 
central galaxies/subhalos in the $\mathtt{TreePM}$ simulation. Since 
we want our model to reproduce observations, our selection is based 
on $f^\mathrm{cen}_\mathrm{SFS}(M_*)$, the fraction of central galaxies 
within the SFS measured from the SDSS DR7 VAGC (Section~\ref{sec:sdss}). 
Below, we describe how we derive $f^\mathrm{cen}_\mathrm{SFS}(M_*)$ and 
use it to select star-forming central galaxies in our model. Afterwards 
we describe how we initialize the SFRs and $M_*$ of these galaxies.

Often in the literature, an empirical color-color or SFR--$M_*$ cut 
that separates the two main modes (red/blue or star-forming/quiescent) 
in the distribution is chosen to classify 
galaxies~\citep[\emph{e.g.}][]{baldry2006, blanton2009, drory2009, peng2010, moustakas2013, hahn2015}.
The red/quiescent or blue/star-forming fractions derived from this sort 
of classification, by construction, depend on the choice of cut and 
neglect galaxy subpopulations such as transitioning galaxies~\emph{i.e.} 
galaxies in the ``green valley''. Instead, for our $f^\mathrm{cen}_\mathrm{SFS}(M_*)$, 
we use the SFS identified from the \cite{hahn2018a} method. \cite{hahn2018a}~uses Gaussian 
Mixture Models and the Bayesian Information Criteria in order to fit the 
SFR--$M_*$ relation of a galaxy population and identify its SFS. This 
data-driven approach relaxes many of the assumptions and hard cuts that 
go into other methods. Furthermore, \cite{hahn2018a} demonstrate its method can 
be flexibly applied to a wide range of SFRs and $M_*$s and for multiple 
simulations. The weight of the SFS GMM component from the method provides 
an estimate of $f^\mathrm{cen}_\mathrm{SFS}$. In the right panel of 
Figure~\ref{fig:groupcat}, we present the SSFR distribution, $p(\log \mathrm{SSFR})$,
of the SDSS DR7 central galaxies within $10.6 < \log M_* < 10.8$ with 
the SFS GMM component shaded in blue. 
%plot the SFS  GMM component (blue shaded region) of the $p(\log \mathrm{SSFR})$  for the SDSS DR7 central galaxies within $10.6 < \log M_* < 10.8$. 
The SFS constitutes $f^\mathrm{cen}_\mathrm{SFS} = 0.21$ of the SDSS 
central galaxies in this stellar mass bin. Using the $f^\mathrm{cen}_\mathrm{SFS}$ 
estimates, we fit $f^\mathrm{cen}_\mathrm{SFS}$ as a linear function of 
$\log M_*$ similar to \cite{wetzel2013,hahn2017b}: 
\beq \label{eq:f_cen_sfms}
f^\mathrm{cen}_\mathrm{SFS, bestfit}(M_*) = -0.627\,(\log\,M_* - 10.5) + 0.354. 
\eeq
We note that this is in good agreement with the $f_\mathrm{Q}^\mathrm{cen}(M_*; z \sim 0)$ 
fit from \cite{hahn2017b}. 

For each central subhalo in our simulation, we assign a probability of it 
being on the SFS, using Eq.~\ref{eq:f_cen_sfms} with $M_*$ at $z \sim 0$ 
assigned through subhalo abundance matching (SHAM) to $M_\mathrm{peak}$, 
the maxmum host halo mass that it ever had as a central subhalo~\citep{conroy2006,vale2006,yang2009,wetzel2012,leja2013,wetzel2013,wetzel2014,hahn2017b}. 
SHAM, in its simplest form, assumes a one-to-one mapping between subhalo 
$M_\mathrm{peak}$ and galaxy stellar mass, $M_*$, that preserves rank 
order: $n({>}M_\mathrm{peak}) > n({>}M_*)$. In practice, we apply a $0.2$ 
dex log-normal scatter in $M_*$ at fixed $M_\mathrm{peak}$ based on the 
observed SHMR. For $n({>}M_*)$, we use observed stellar mass function (SMF) 
from \cite{li2009}, which is based on the same SDSS NYU-VAGC sample as our 
group catalog. Based on the probabilities from Eq.~\ref{eq:f_cen_sfms} and 
SHAM $M_*$, we randomly identify centrals from our simulation as SF at 
$z \sim 0$. \cite{tinker2017b,tinker2018} found that quenching is 
independent of halo growth rate and therefore we randomly select SF subhalos.  
In our model, we assume that once a SF galaxy quenches its star formation, 
it remains quiescent.  %\todo{maybe something about rejuvenation?} 
Without any quiescent galaxies rejuvenating their star formation, galaxies
on the SFS at $z\sim0$ are also on the SFS at $z > 0$. Using this assumption
the SF centrals we select at $z \sim 0$ are also on the SFS at the intial
redshift of our model: $z \sim 1$. 

We next initialize the SF centrals at $z\sim1$ using the observed SFR-$M_*$ 
relation of the SFS with $M_*$ assigned using SHAM with a SMF interpolated
between the \cite{li2009} SMF and the SMF from \cite{marchesini2009} at 
$z = 1.6$. We choose the \cite{marchesini2009} SMF, among others, because it 
produces interpolated SMFs that monotonically increase at $z < 1$. As noted 
in \cite{hahn2017b}, at $z \approx 1$, the SMF interpolated between the 
\cite{li2009} and \cite{marchesini2009} SMFs is consistent with more recent 
measurements from \cite{muzzin2013} and \cite{ilbert2013}. Observations in 
the literature at $z \sim 1$, however, not only use galaxy properties 
derived differently from the SDSS VAGC but they also find SFS with significant 
discrepancies from one another. \cite{speagle2014} compiles the SFR-$M_*$ 
relation of the SFS from 25 studies in the literature, each with different 
methods of deriving galaxy properties. Even \emph{after} their calibration, 
for a fixed $M_* = 10^{10.5}\, M_\odot$, the SFRs of the SFSs at $z \sim 1$ 
vary by more than a factor of 2~\citep[see Figure 2 of][]{speagle2014}. 
With little consensus on the SFS at $z\sim1$, and consequently its redshift
evolution, we flexibly parameterize the SFS SFR, 
$\log\mathrm{SFR}_\mathrm{SFS}(M_*, z)$, 
with free parameters $m^\mathrm{low}_{M_*}$, $m^\mathrm{high}_{M_*}$, and 
$m_z$. These parameters characterize the stellar mass dependence of the SFS 
below and above $10^{10} M_\sun$ and its redshift dependence, respectively. 

We parameterize the $\log\mathrm{SFR}$ of the SFS as, 
\beq \label{eq:logsfr_ms}
\logsfr(M_*, z) =  \begin{cases}
m^\mathrm{low}_{M_*}\,(\log M_* - 10.)  & \text{for}\,M_* < 10^{10}M_\sun \\
m^\mathrm{high}_{M_*}\,(\log M_* - 10.)  & \text{for}\,M_* \geq 10^{10}M_\sun.
\end{cases}
+ m_z (z - 0.05) - 0.19.
\eeq
We assign SFRs to our SF centrals at $z\sim1$ by sampling a log-normal 
distribution centered about $\log\,\mathrm{SFR}_\mathrm{MS}(M_*, z=1)$ 
with a constant scatter of $0.3\,\mathrm{dex}$, motivated from 
observations~\citep{daddi2007, noeske2007, magdis2012, whitaker2012}.
Later when comparing to observations, we choose conservative priors 
for the parameters $m^\mathrm{low}_{M_*}$, $m^\mathrm{high}_{M_*}$ and $m_z$
that encompass the best-fit SFS from~\cite{speagle2014} and measurements 
from~\cite{moustakas2013} and~\cite{lee2015}. With our SF centrals initalized 
at $z \sim 1$, next, we describe how we evolve their SFR and $M_*$.

%%%%%%%%%%%%%%%%%%%%%%%%%%%%%%%%%%%%%
% Figure 3 
%%%%%%%%%%%%%%%%%%%%%%%%%%%%%%%%%%%%%
\begin{figure}
\begin{center}
\includegraphics[width=0.5\textwidth]{figs/illustris_sfh.pdf} 
    \caption{Galaxies in the Illustris simulation have SFHs that evolve along the
    SFS, with their SFRs stochastically fluctuating about the $\logsfr$ of the SFS.
    We highlight $\Delta \logsfr$, SFR with respect to $\logsfrsfs$ (Eq.~\ref{eq:logsfr_sf}), 
    for a handful of galaxies with $10^{10.5}< M_* < 10^{10.6}M_\odot$ at $z\sim0$. 
    We calculate $\Delta \logsfr$ with $\logsfrsfs$ identified using the \cite{hahn2018a} 
    method, same as in Section~\ref{sec:sfcen}. The implementation of $\mathrm{SFR}$
    variability in the SFHs of star-forming centrals in our model 
    (Section~\ref{sec:modelevol}) is motivated by the SFHs of Illustris galaxies above.}
\label{fig:illsfh}
\end{center}
\end{figure}
%%%%%%%%%%%%%%%%%%%%%%%%%%%%%%%%%%%%%

%%%%%%%%%%%%%%%%%%%%%%%%%%%%%%%%%%%%%
% Figure 4 
%%%%%%%%%%%%%%%%%%%%%%%%%%%%%%%%%%%%%
\begin{figure}
\begin{center}
\includegraphics[width=0.85\textwidth]{figs/sfh_pedagogical.pdf}
    \caption{We incorporate star formation variability in our model using a 
    ``star formation duty cycle'' where the SFRs of SF centrals fluctuate about 
    the $\logsfrsfs$ on some timescale $t_\mathrm{duty}$. In our fiducial 
    prescription, we randomly sample $\Delta \logsfr$ from a log-normal 
    distribution with $0.3$ dex scatter at each timestep. We illustrate 
    $\Delta\logsfr_i(t)$ of two SF centrals with star formation duty cycles 
    on $t_\mathrm{duty} = 1$ Gyr (blue) and $5$ Gyr (orange) timescales 
    in the left panel. $\Delta \logsfr(t)$ determines the SFH and hence 
    the $M_*$ growth of the SF central galaxies (Eq.~\ref{eq:integ_mass}).  
    On the right, we illustrate the $\mathrm{SFR}$ and $M_*$ evolutions 
    of the corresponding SF centrals. For reference, we include 
    $\logsfrsfs(M_{*,i}(t), t)$ that the galaxies' SFR and $M_*$ evolve along 
    (black solid). We also include $\logsfrsfs(M_*)$ at various redshifts 
    between $z = 1$ to $0.05$ (dotted  lines). \emph{The star-forming 
    centrals in our model evolve their SFRs and $M_*$ along the SFS 
    with their SFRs fluctuate about $\logsfrsfs$}.} \label{fig:sfh_model}
\end{center}
\end{figure}
%%%%%%%%%%%%%%%%%%%%%%%%%%%%%%%%%%%%%


\subsection{Evolving along the Star Formation Sequence} \label{sec:modelevol} 
The tight correlation between the SFRs and stellar masses of star-forming 
galaxies on the SFS has been observed spanning over four orders of magnitude 
in stellar mass, with a roughly constant scatter of ${\sim}0.3$ dex, and out 
to $z > 2$ 
(\emph{e.g.}~\citealt{noeske2007,daddi2007,elbaz2007,salim2007,santini2009,karim2011,whitaker2012,moustakas2013,lee2015}; see also references in \citealt{speagle2014}). 
This correlation is also predicted by modern galaxy formation models~\citep[][see 
\citealt{hahn2018a} and references therein]{somerville2015}. The SFS 
naturally presents itself as an anchoring relationship to characterize 
the star formation and $M_*$ growth histories of SF galaxies throughout $z < 1$. More 
specifically, \emph{we characterize the SFH of each star-forming central 
with respect to the $\logsfr$ of the SFS} (Eq.~\ref{eq:logsfr_ms}):
\beq \label{eq:logsfr_sf} 
\logsfr_i(M_*, t) = \logsfrsfs(M_*, t) + \Delta\logsfr_i(t).
\eeq
Since SFHs determine the $M_*$ growth of galaxies, in this prescription, 
$\Delta \logsfr_i(t)$ dictates the SFH and $M_*$ evolution a star-forming 
central. Next, we describe our prescription for $\Delta \logsfr$.

One simple prescription for $\Delta \logsfr(t)$ would be to keep $\Delta \logsfr$ 
fixed throughout $z < 1$ to the offsets from the $\logsfrsfs$ in the 
initial SFRs of our SF centrals at $z\sim1$ similar to simple analytic 
models such as \cite{mitra2015}. Galaxies with higher than average 
initial SFRs continue evolving above the average SFS, while SF centrals 
with lower than average initial SFRs continue evolving below the average 
SFS. In addition to not being able to reproduce observations, which we
later demonstrate, we also do not find such SFHs in SF galaxies of 
hydrodynamic simulations such as Illustris~\citep{vogelsberger2014,genel2014}. 
In Figure~\ref{fig:illsfh}, we plot $\Delta \logsfr_i$ of star-forming 
galaxies in the Illustris simulation as a function of cosmic time. These 
galaxies have stellar masses within $10^{10.5}-10^{10.6}M_\odot$ at $z=0$. 
At each simulation output, we calculate $\Delta \logsfr$ using Eq.~\ref{eq:logsfr_sf}
with $\logsfrsfs$ derived from the SFS identified using the same \cite{hahn2018a}
method as in Section~\ref{sec:sfcen}. As the highlighted $\Delta \logsfr_i$
illustrate, star-forming galaxies in Illustris evolve along the SFS, with
their SFRs fluctuating about $\logsfrsfs$.

Motivated by the SFHs of Illustris SF galaxies, we introduce variability 
to the SFHs of our SF centrals in the form of a ``\emph{star formation duty cycle}''. 
Within the Eq.~\ref{eq:logsfr_sf} SFH, we parameterize $\Delta \logsfr_i$ to 
fluctuate about the $\logsfrsfs$ on timescale, $t_\mathrm{duty}$, with 
amplitude sampled to reproduce the observed log-normal distribution with 
$0.3\,\mathrm{dex}$ scatter at every timestep. For our fiducial star 
formation duty cycle prescription, we randomly sample $\Delta \logsfr_i$ 
from a log-normal distribution with $\sigma = 0.3\,\mathrm{dex}$. We 
illustrate $\Delta \logsfr_i(t)$ of SF centrals with our star formation 
duty cycle prescription on $t_\mathrm{duty}=1$ Gyr (blue) and $5$ Gyr (orange) 
timescales in the left panel of Figure~\ref{fig:sfh_model}. 
The shaded region represents the observed $0.3\,\mathrm{dex}$ $1\sigma$ 
scatter of $\logsfr$ in the SFS. This $\Delta \logsfr$ prescription by 
construction reproduces the observed log-normal SFR distribution of the SFS 
at any point in the model. Although, we do not expect this simplified 
prescription to reflect the individual SFHs of SF centrals, 
we seek to statistically capture the stochasticity from gas accretion, 
star-bursts, and feedback mechanisms for the entire SF population. 
Measuring $t_\mathrm{duty}$ in the duty cycle parameterization
provides us with an estimate of the timescale of such star formation 
variabilities and thus provide additional useful constraints on the 
physics of galaxy formation. 

Using our fiducial SFH prescription, we evolve both the SFR and $M_*$ 
of our SF centrals along the SFS. Based on Eq.~\ref{eq:logsfr_sf},
the SFRs of our SF centrals are functions of $M_*$. Meanwhile, $M_*$ 
is the integral of the SFR over time: 
\beq \label{eq:integ_mass} 
M_*(t) = f_\mathrm{retain} \int\limits_{t_0}^{t} \mathrm{SFR(M_*, t)}\,\mathrm{d}t + M_0. 
\eeq
$t_0$ and $M_0$ are the initial cosmic time and stellar mass at $z \sim 1$, 
respectively. $f_\mathrm{retain}$ here is the fraction of stellar mass 
that is retained after supernovae and stellar winds; we use 
$f_\mathrm{retain} = 0.6$~\citep{wetzel2013}. We can now evolve the SFR and 
$M_*$ of our star-forming centrals until the final $z=0.05$ snapshot by 
solving the differential equation of Eqs.~\ref{eq:logsfr_sf} and~\ref{eq:integ_mass}. 
On the right panel of Figure~\ref{fig:sfh_model}, we present the 
$\mathrm{SFR}$ and $M_*$ evolutions of two SF centrals with 
$t_\mathrm{duty}=1$ Gyr (blue) and $5$ Gyr (orange), same 
as the left panel. For reference, we include the mean $\logsfr$ of the SFS 
that the galaxies' SFR and $M_*$ evolve along, $\logsfrsfs(M_{*,i}(t), t)$ 
(black solid). We also include $\logsfrsfs(M_*)$ (dotted lines) at various 
redshifts between $z = 1$ to $0.05$. Based on the SFH prescription in our 
model, star-forming centrals evolve their SFRs and $M_*$ along the SFS 
with their SFRs fluctuate about $\logsfrsfs$.

With the evolved SFRs and $M_*$ of SF centrals, we can compare our model 
to observed galaxy population scatistics such as the quiescent fraction 
and SMF to constrain its free parameters. Our model, run with these 
inferred parameters, then reproduces observations of galaxy properties 
and has host halo properties from the $\mathtt{TreePM}$ $N$-body simulation. 
This allows us to compare our models to observations of the galaxy-halo 
connection such as the SHMR. In the following section, we present this 
comparison and the constraints we derive on the role and timescale of 
star formation variability in SF central galaxies.

%%%%%%%%%%%%%%%%%%%%%%%%%%%%%%%%%%%%%
% Figure 4 
%%%%%%%%%%%%%%%%%%%%%%%%%%%%%%%%%%%%%
\begin{figure}
\begin{center}
\includegraphics[width=\textwidth]{figs/qaplot_abc.pdf}
    \caption{Our models with different star formation duty cycle timescales 
    (blue: $t_\mathrm{duty}{=}1\,\mathrm{Gyr}$; red: $t_\mathrm{duty}{=}10\,\mathrm{Gyr}$) 
    run with median values of their ABC posterior distribution have SMFs and SFSs consistent 
    with observations (left and middle). \emph{They however produce significantly different 
    scatter in $\log\,M_*$ at fixed $\log\,M_\mathrm{halo}$ --- scatter in the SHMR (right)}. 
    By comparing the scatter in SHMR of our models to observational constraints on the SHMR, 
    we constrain the timescale of the star formation duty cycle and thereby the SFHs of star 
    forming galaxies.
    }
\label{fig:abc_demo}
\end{center}
\end{figure}
%%%%%%%%%%%%%%%%%%%%%%%%%%%%%%%%%%%%%

\section{Results} \label{sec:results} 
Our model takes $\mathtt{TreePM}$ central subhalos and tracks their SFR 
and $M_*$ evolution using a flexible parameterization of the SFS and 
SFHs that incorporate variability through a star formation duty cycle.
At $z = 0.05$, its final timestep, our model predicts properties such 
as SFR, $M_*$, and host halo mass, $M_h$, of central galaxies it 
classifies as star-forming. We now use these resulting properties to compare our 
model to observations and constrain its free parameters--- the 
SFS parameters of Eq.~\ref{eq:logsfr_ms}. Since the focus of our model 
and this paper is SF centrals, the main observable we use is the SMF of 
SF centrals in SDSS, which we estimate as 
\beq \label{eq:smf_sf_cen} 
\Phi^\mathrm{SDSS}_\mathrm{SF,cen} = f^\mathrm{cen}_\mathrm{SFS} \times f_\mathrm{cen} \times \Phi^\mathrm{Li\&White(2009)}.
\eeq
$f^\mathrm{cen}_\mathrm{SFS}$ is the fraction of central galaxies on the 
SFS, which we fit in Eq.~\ref{eq:f_cen_sfms}. $f_\mathrm{cen}$ is the 
central galaxy fraction from \cite{wetzel2013} and $\Phi^\mathrm{Li\&White(2009)}$ 
is the SMF of the SDSS from \cite{li2009}. If our model reproduces the 
observed $\Phi^\mathrm{SDSS}_\mathrm{SF,cen}$, by construction it reproduces 
the observed quiescent fraction. 

For the comparison between our model to observation, we use the parameter 
estimation framework of Approximate Bayesian Computation (ABC). ABC has the 
advantage over standard approaches to parameter inference in that it does not 
require evaluating the likelihood. It relies only on a simulation of the observed 
data and a distance metric to quantify the ``closeness'' between the observed data
and simulation. Many variations of ABC has been used in astronomy and 
cosmology~\citep[\emph{e.g.}][]{cameron2012,weyant2013,ishida2015,alsing2018}. 
We use ABC in conjunction with the efficient Population Monte Carlo (PMC)
importance sampling as in~\citep{hahn2017b, hahn2017a}. For initial sampling 
of our ABC particles, \emph{i.e.} the priors of our parameters 
$m^\mathrm{low}_{M_*}$, $m^\mathrm{high}_{M_*}$, and $m_z$, we use uniform 
distributions over the ranges $[0.0, 0.8]$, $[0.0, 0.8]$, and 
$[0.5, 2.]$, respectively. As we mention earlier in Section~\ref{sec:sfcen}, 
the range of the prior were conservatively chosen to encompass the best-fit 
SFS from~\cite{speagle2014} 
and measurements from~\cite{moustakas2013} and~\cite{lee2015} at $z \sim 1$. 
Finally, for our distance metric we use the following distance between 
the SMF of the star-forming centrals in our model to the observed 
$\Phi^\mathrm{SDSS}_\mathrm{SF,cen}$: 
\beq
\rho_\Phi = \sum\limits_{M} \left( \frac{\Phi^\mathrm{sim} - \Phi^\mathrm{SDSS}_\mathrm{SF,cen}}{\sigma'_\Phi}\right)^2.
\eeq
$\Phi^\mathrm{sim}(M)$ above is the SMF of the SF centrals in our model 
and $\sigma'_\Phi(M)$ is the uncertainty of $\Phi^\mathrm{SDSS}_\mathrm{SF,cen}$, 
which we derive by scaling the~\cite{li2009} uncertainty of $\Phi^\mathrm{SDSS}$ 
derived from mock catalogs. %\todo{We emphasize that ABC is not very sensitive to the distance metric.}
For the rest of our ABC-PMC implementation, we strictly follow the implementation 
of~\cite{hahn2017a} and \cite{hahn2017b}. We refer reader to those papers for 
further details.

%%%%%%%%%%%%%%%%%%%%%%%%%%%%%%%%%%%%%
% Figure 5 
%%%%%%%%%%%%%%%%%%%%%%%%%%%%%%%%%%%%%
\begin{figure}
\begin{center}
\includegraphics[width=0.45\textwidth]{figs/SHMRscatter_tduty.pdf}
    \caption{With shorter star formation duty cycle timescales, $t_\mathrm{duty}$, our 
    model produces smaller scatter in $\log M_*$ at $M_h{=}10^{12} M_\odot$, 
    $\sigma_{\log\,M_*}$. For $t_\mathrm{duty}$ ranging from $10$ to $0.5\,\mathrm{Gyr}$, 
    $\sigma_{\log\,M_*}$ ranges from {\ch $0.32\substack{+0.019\\ -0.021}$ to 
    $0.26\substack{+0.010\\-0.012}$}. For comparison, we include observational 
    $\sigma_{\log\,M_*}$ constraints from \cite{yang2009, more2011, leauthaud2012, tinker2013, zu2015} 
    (see Section~\ref{sec:sfdutycycle}). Star formation varability on short 
    timescales is necessary to produce a tight SHMR roughly consistent with 
    constraints from observations. 
    }
\label{fig:sigMstar_duty}
\end{center}
\end{figure}
%%%%%%%%%%%%%%%%%%%%%%%%%%%%%%%%%%%%%

\subsection{Timescale of the Star Formation Duty Cycle} \label{sec:sfdutycycle}
We present the SMFs (left), SFSs (center), and the scatter in $\log M_*$ at a 
given $\log M_h$, $\sigma_{\log\,M_*}$, (right) of our model run using SFHs with 
two different duty cycle timescales, $t_\mathrm{duty} = 10\,\mathrm{Gyr}$ (red) 
and $1\,\mathrm{Gyr}$ (blue), in Figure~\ref{fig:abc_demo}. For each 
$t_\mathrm{duty}$, we evaluate our model at the median parameter values of the respective posterior distributions derived 
using ABC. For both $t_\mathrm{duty}$, our model successfully produces SMFs, 
$\Phi^\mathrm{SDSS}_\mathrm{SF,cen}$, and SFSs consistent with observations, 
as expected (left and center panels). Despite their consistency with observations, 
however, different $t_{\rm duty}$ of the models predict significantly different 
$\sigma_{\log\,M_*}$, particularly below $M_h < 10^{12.5}M_\odot$. We further
illustrate the sensitivity of $\sigma_{\log\,M_*}$ of our model to $t_{\rm duty}$
in Figure~\ref{fig:sigMstar_duty}, where we present $\sigma_{\log\,M_*}$ at
fixed halo mass $M_h = 10^{12} M_\odot$ as a function of $t_{\rm duty}$. As in 
Figure~\ref{fig:abc_demo}, $\sigma_{\log\,M_*}$ at each $t_{\rm duty}$ is predicted 
from our model run with $t_{\rm duty}$ and $\theta$ from the corresponding ABC 
posterior distribution. The shaded width represents the $68\%$ and $95\%$ 
confidence intervals. For $t_\mathrm{duty}$ ranging from $10$ to $0.5\,\mathrm{Gyr}$,
$\sigma_{\log\,M_*}$ ranges from 
{\color{red}
$0.32\substack{+0.019\\ -0.021}$ to $0.26\substack{+0.010\\-0.012}$ 
}
--- a shorter star formation duty cycle timescale produces significantly 
smaller scatter in the SHMR. Hence, \emph{observational constraints on 
the SHMR can be used to constrain the timescale of star formation variability 
and SFH of SF central galaxies.} 

We compare $\sigma_{\log M_*}$ predicted from our model to observational 
constraints in the literature. These constraints are mainly derived from 
fitting some sort of halo occupation based models to observations of galaxy 
clustering, SMF, satellite kinematics, or galaxy-galaxy weak lensing. In 
Figure~\ref{fig:sigMstar_duty}, we include $\sigma_{\log M_*}$ constraints 
from~\cite{more2011, leauthaud2012, tinker2013, zu2015} and Cao et al. (in preparation).
\cite{more2011} and \cite{zu2015} use SDSS VAGC to fit satellite kinematics 
and galaxy clustering and galaxy-galaxy lensing measurements, repsectively.
Meanwhile, \cite{leauthaud2012, tinker2013} use COSMOS to fit the SMF, galaxy 
clustering, and galaxy-galaxy lensing. Finally, Cao et al. (in preparation) 
uses the kurtosis of the line-of-sight pairwise velocity dispersion between 
central galaxies and all neighboring galaxies to constrain the scatter in SHMR 
at low halo masses. We note that \cite{leauthaud2012, zu2015} measure 
$\sigma_{\log M_*}$ for all central galaxies, not only star-forming. At 
$M_h\sim 10^{12}M_\odot$, however, \cite{more2011, tinker2013} find only 
only a $< 1\sigma$ difference in $\sigma_{\log M_*}$ between SF and quiescent 
centrals, so we include their measurements in our comparison. 

\begin{comment}
    \bitem
        \item \cite{yang2009} uses a galaxy group catalog constructed from SDSS DR4 to 
            directly measure the log-normal distribution of $M_*$ at a given $M_h$.
            For blue centrals they find $\sigma_{\log\,M_*}(M_h \sim 10^{12.16}M_\odot) = 0.122 \pm 0.03$ 
            averaged over two SDSS samples and two group mass difference.

        \item \cite{more2011} uses a halo occupation based model to fit the stacked 
            velocity dispersion of satellite galaxies as a function of $M_*$ in SDSS VAGC.  
            One of the free parameters in their model is $\sigma_{\log\,M*}$, constant
            across $M_h$, for blue centrals. The best-fit model also predicts 
            $\sigma_{\log\,M_h}$ as a function $M_*$ for blue centrals. 
            
        \item \cite{leauthaud2012} joint analysis of galaxy–galaxy weak lensing, galaxy 
            spatial clustering, and galaxy number densities of the COSMOS data. 
            $\sigma_{\log\,M*}$ is constant. They get the constraint 
            $\sigma_{\log\,M_*}(M_h \sim 10^{12}M_\odot) = 0.206 \pm 0.031$ at 
            $0.2 < z < 0.48$ for all galaxies.

        \item \cite{tinker2013} uses the stellar mass function, galaxy clustering, and 
            galaxy–galaxy lensing within the COSMOS survey to constrain the stellar-to-halo 
            mass relation (SHMR) of star forming and quiescent galaxies over the redshift 
            range $z = [0.2, 1.0]$. They constrain the constant 
            $\sigma_{\log\,M_*}(M_h \sim 10^{12}M_\odot) = 0.21 \pm 0.06$ at 
            $0.22 < z < 0.48$ for star-forming galaxies.

        %\item \cite{reddick2013} constrain a model that abundance matches galaxies with the 
        %    peak circular velocity of their halos with the SDSS NYU-VAGC projected two-point 
        %    galaxy clustering and the observed conditional stellar mass functions. The model
        %    that best reproduces the data has a scatter of $0.20 \pm 0.03$ dex for all 
        %    galaxies.

        \item \cite{zu2015} 
            $\mathtt{iHOD}$: an exteded HOD model, to fit the galaxy clustering and 
            the galaxy-galaxy lensing measured from SDSS main galaxy sample and NYU-VAGC. 
            The best-fit model has $\sigma_{\log\,M_*}(M_h \sim 10^{12}M_\odot) = 0.22 \pm 0.02$ 
            for all galaxies. 
    \eitem
\end{comment}

%On the right panel, we include $\sigma_{\log\,M_h}$ constraints from \cite{more2011}
%along with constraints from \cite{mandelbaum2006a, conroy2007, velander2014, han2015}. 
%\cite{mandelbaum2006a} and \cite{velander2014} use halo occupation models to 
%analyze galaxy-galaxy weak lensing measurements of SDSS and the 
%Canada-France-Hawaii Telescope Lensing Survey~\citep[CFHTLenS;][]{heymans2012}, 
%respectively. \cite{conroy2007}, similar to \cite{more2011}, use a halo model 
%to fit the stacked satellite velocity dispersions for SDSS VAGC. Lastly, 
%\cite{han2015} use a maximum likelihood weak lensing analysis to fit shapes 
%of SDSS source galaxies individually (unlike the other stacked weak lensing
%analyses) for the G3Cv5 GAMA group catalog~\citep{robotham2011}. From these 
%works, we use $\sigma_{\log M_h}$ from the uncertainties in their $M_h$ 
%measurements at $M_*{\sim}10^{10}M_\odot$. From 
%\cite{mandelbaum2006a, more2011, velander2014} we have $\sigma_{\log\,M_h}$ 
%for blue/late-type/star forming centrals while we have $\sigma_{\log\,M_h}$ 
%for all centrals from \cite{conroy2007, han2015}. 

\begin{comment}
    \bitem
        \item \cite{mandelbaum2006a} use halo occupation based model to fit SDSS galaxy-galaxy 
            weak lensing in $M_*$ bins in order to constrain the distribution of $M_h$ of 
            early/late-type centrals. 

        \item \cite{conroy2007} use halo occupation based model to fit the stacked satellite velocity
            disperions for SDSS VAGC in bins of $M_*$. For each $M_*$ bin they measure $M_h$ for 
            all central galaxies. We can convert this measurement to an upper limit on 
            $\sigma_{\log\,M_h}$.  

        %\item \cite{vanuiter2011}: weak lensing from imaging data from the $\sim 300\,\mathrm{deg}^2$ 
        %    overalp between the second Red Sequence Cluster Survey (RCS2) and SDSS DR7 using halo model 
        %    that accounts for the clustering of the lenses and distinguishes between satellite
        %    and centrals.  At $\log\,M_* = [10.0-10.5]$ $M_\mathrm{halo} = 0.56\substack{+1.66\\ -0.55}\times 10^{11}h^{-1} M_\odot$. 

        \item \cite{velander2014} conducts a stacked galaxy-galaxy weak lensing analysis
            using a HOD model on CFHTLenS data. They measure $M_h$ in bins of $M_*$ for blue/red 
            centrals at $z \sim 0.3$. We note that they correct for the scatter added from 
            the $M_*$ binning due to $M_*$ uncertainties using simulated lens and source 
            catalogs. At $< M_*> = 0.54 \times 10^10 M_\odot/h^2$ 
            $M_\mathrm{halo} = 2\substack{+0.64\\ -0.62}\times 10^{11}h^{-1} M_\odot$. 

        \item \cite{han2015} uses a maximum likelihood weak lensing analysis to fits
            shape of SDSS source galaxies individually and derive $M_h$ measurements 
            of the G3Cv5 GAMA group catalog in bins of central galaxy $M_*$. Unlike 
            the other weak lensing measurements, they do not use stacked lensing 
            measurements or a HOD model to derive the SHMR.
    \eitem
\end{comment}

%We find that by decreasing the timescale of stochasticity on a simple SFH model that traces the overall 
%SFS evolution does in fact decrease the scatter seen in the SMHMR. 
%The $\sigma_{\log\,M_h}$ constraints from the literature are significantly scatter from $0.12$ to $0.6\,\mathrm{dex}$, right panel of Figure~\ref{fig:sigMstar_duty}.  $\sigma_{\log\,M_h}$ from our model, regardless of $t_\mathrm{duty}$ is well within this range, so the comparison does not provide much constraint on $t_\mathrm{duty}$. On the other hand, 
Besides Cao et al. (in preparation), the $\sigma_{\log\,M_*}$ constraints 
in the literature are relatively consistent: $\sigma_{\log\,M_*} \lesssim 0.2$.
However, it's important to note that these constraints are derived using halo
models that fix $\sigma_{\log\,M_*}$, independent of $M_h$. As constraining 
power comes from a wide range of halo masses, such analyses are not ideal 
measurements of $\sigma_{\log\,M_*}$ at a specific $M_h\sim 10^{12}M_\odot$. 
{\ch something on the discrepancy}. 
These constraints are likely overestimates of the intrinsic $\sigma_{\log\,M_*}$,
since observational constraints measure the quadratic sum of the intrinsic scatter 
and measurement scatter. Hence, a longer duty cycle timescale predicts $\sigma_{\log\,M_*}$
in tension with observations. 
%it's clear from the comparison that, with a longer duty cycle timescale, our mode produces scatters too large to reconcile with observations.
\emph{A short duty cycle timescale--- $t_\mathrm{duty} \lesssim 0.5\,\mathrm{Gyr}$--- 
is necessary for our model to ameliorate the tension with $\sigma_{\log\,M_*}$ 
constraints in the literature}. This $t_\mathrm{duty}$ constraint implies
that SFRs of star-forming galaxies have variations on 
$\lesssim 0.5\,\mathrm{Gyr}$ timescales. However, even with the shortest 
duty cycle timescale we probe  
{\color{red}
($\sigma_{\log\,M_*}{=}\,0.26\substack{+0.010\\ -0.012}$)
}
our model predicts $\sigma_{\log M_*}$ larger than in observations. 

%%%%%%%%%%%%%%%%%%%%%%%%%%%%%%%%%%%%%
% Figure 6 
%%%%%%%%%%%%%%%%%%%%%%%%%%%%%%%%%%%%%
\begin{figure}
\begin{center}
\includegraphics[width=0.5\textwidth]{figs/Mhacc_dSFR.pdf}
\caption{We incorporate assembly bias into the SF centrals of our model by 
    correlating host halo accretion history with the SFH with respect to the 
    SFS SFR. We plot the relative halo accretion history, $M_h(t)/M_h(z{=}0.05)$ 
    for two arbitrarily chosen SF centrals with $M_h(z{=}0.05)\sim10^{12}M_\odot$, 
    in the top panel. In the two panels below we present $\Delta\log\,\mathrm{SFR}$, 
    SFH with respect to the SFS, of these galaxies for our model with correlation 
    coefficients $r=0.5$ and $0.99$ (middle and bottom). At some $t$ (dotted), 
    $\Delta\log\,\mathrm{SFR}(t)$ is correlated with halo accretion over the 
    range $t - t_\mathrm{dyn}$ to $t_\mathrm{dyn}$ (shaded top panel). The 
    SFHs illustrate how $\Delta\log\,\mathrm{SFR}(t)$ correlates with 
    $\Delta M_h = M_h(t) - M_h(t-t_\mathrm{dyn})$ and with stronger correlations 
    for larger $r$.}
\label{fig:mhacc_dsfr}
\end{center}
\end{figure}
%%%%%%%%%%%%%%%%%%%%%%%%%%%%%%%%%%%%%

%%%%%%%%%%%%%%%%%%%%%%%%%%%%%%%%%%%%%
% Figure 7 
%%%%%%%%%%%%%%%%%%%%%%%%%%%%%%%%%%%%%
\begin{figure}
\begin{center}
\includegraphics[width=0.75\textwidth]{figs/SHMRscatter_tduty_abias2.pdf}
    \caption{When we incorporate assembly bias, $\sigma_{\log\,M_*}$ of our 
    model spans the various observational constraints and predictions from 
    galaxy formation models. We plot $\sigma_{\log\,M_*}(M_h=10^{12}M_\odot)$ 
    as a function of the star formation duty cycle timescale, $t_\mathrm{duty}$,
    for our best-fit model with no assembly bias (blue) and with correlation 
    coefficients between host halo accretion history and SFH of $r{=}0.5$ 
    (orange) and $0.99$ (green).  Stronger assembly bias at 
    $t_\mathrm{duty}{<}5\,\mathrm{Gyr}$ significantly reduces the scatter 
    in SHMR, $\sigma_{\log\,M_*}$. In the left panel, we include observational 
    cosntraints from Figure~\ref{fig:sigMstar_duty}; in the right, we include 
    predictions from hydrodyanmical simulations (dotted region), semi-analytic
    models (hatched), and the {\sc UniverseMachine}~\citep{behroozi2018a} 
    (dashed). We also include on the right $\sigma_{\log\,M_*}$ derived from 
    matching \cite{abramson2016} SFHs to halos via abundance matching.
    With strong assembly bias and $t_\mathrm{duty} < 1\,\mathrm{Gyr}$, our 
    model conservatively agrees with $\sigma_{\log\,M_*}$ observational 
    cosntraints and  predictions from hydrodynamic simulations. 
    }
\label{fig:sigMstar_duty_abias}
\end{center}
\end{figure}
%%%%%%%%%%%%%%%%%%%%%%%%%%%%%%%%%%%%%
\subsection{The Need for Assembly Bias?}
A shorter star formation duty cycle timescale produces smaller scatter in the 
SHMR of our model. This dependence on the duty cycle timescale, allows us to 
compare our model with measurements of $\sigma_{\log M_*}$ and 
$\sigma_{\log M_h}$ to constrain $t_\mathrm{duty}$, which reflect the star 
formation variability timescale. Such comparisons in the previous section, 
demonstrate that $t_\mathrm{duty} \lesssim 0.5\,\mathrm{Gyr}$ is necessary 
to reduce tensions with observed constraints. Yet a short duty cycle 
timescale alone is not enough to conservatively reproduce observed 
$\sigma_{\log\,M_*}$ constraints. In our fiducial model, the SFR and $M_*$ 
evolution of star-forming centrals are independent from the evolution of 
host halo properties. However, as the lack of evolution in the observed 
SHMR scatter and comparison from the previous section suggest, the SFR 
and $M_*$ evolution of star-forming centrals in our model is likely 
correlated with the accretion histories of their host halos. Therefore, in 
this section, we introduce \emph{assembly bias} into the SFH prescription 
of our model.

Assembly bias, most commonly in the literature, refers to the dependence of the 
spatial distribution of dark matter halos on halo properties besides 
mass~\citep{gao2005,wechsler2006,gao2007,wetzel2007,li2008,sunayama2016}.
At low halo mass, older and more concentrated halos form in high density environments. 
While at high halo mass, the effect is the opposite --- younger, less concentrated 
halos form in high-density regions. Furthermore, as predicted by semi-analytic 
models~\citep{croton2007} and found using galaxy group 
catalogs~\citep{yang2006,wang2008,tinker2011,wang2013,lacerna2014,tinker2017,tinker2017b,tinker2018},
this assembly bias propagates beyond spatial clustering to galaxy properties
such as formation histories and star formation properties. For our model, we 
incorporate assembly bias by correlating the SFHs of our star-forming central 
galaxies and their host halo accretion histories with a correlation 
coefficient $r$. 

More specifically, we correlate a galaxy's SFR with respect to the mean SFS 
SFR (\emph{i.e.} $\Delta\log\,\mathrm{SFR}$ in Eq.~\ref{eq:logsfr_sf}) to the 
halo mass accretion over dynamical time. At every $t_\mathrm{duty}$ timestep, 
$t$, $\Delta\logsfr(t)$ is assigned based on 
$\Delta M_h(t) = M_h(t) - M_h(t - t_\mathrm{dyn}(t))$ in $M_\mathrm{max}$ bins 
with a correlation coefficient $r$, a parameter added to our model. Our 
prescription for track halo mass accretion over dynamical time is similar to 
the~\cite{rodriguez-puebla2016,behroozi2018a} empirical models. In 
Figure~\ref{fig:mhacc_dsfr} we illustrate how we incorporate assembly bias into 
our star-forming centrals by correlating the host halo accretion history 
with the SFH with respect to the SFS SFR. We plot the relative halo
accretion histories $M_h(t)/M_h(z{=}0.05)$ of two arbitrarily chosen SF centrals 
with $M_h(z{=}0.05)\sim10^{12}M_\odot$ in the top panel. Below, we plot 
$\Delta\logsfr$, SFH with respect to the SFS, of these galaxies 
for our model with correlation coefficients $r=0.5$ and $0.99$ (middle and bottom). 
At some $t$, we choose a random $\mathtt{TreePM}$ snapshot (dotted), we can see 
that $\Delta\logsfr(t)$ is correlated with halo accretion over the 
period [$t$, $t - t_\mathrm{dyn}$] (shaded top panel). The SFHs illustrate the
correlation between $\Delta\logsfr(t)$ and 
$\Delta M_h = M_h(t) - M_h(t-t_\mathrm{dyn})$ and how $\Delta\logsfr(t)$
correlates more strongly with $\Delta M_h(t)$ for higher $r$ within the 
range $0$ to $1$ (Figure~\ref{fig:mhacc_dsfr}). 

Next, using our implementation of assembly bias, we compare the model with 
different values of $r$ to observations using ABC-PMC (see Section~\ref{sec:sfdutycycle}). 
From the resulting posterior estimates, we examine the scatter in the SHMR 
($\sigma_{\log\,M_*}$) given our model %, which reproduce the observed SMF of SF centrals and SFS,
with $r=0$ (no assembly bias; blue), $0.5$ (orange), and $0.99$ (green) as a 
function of $t_\mathrm{duty}$ in Figure~\ref{fig:sigMstar_duty_abias}. We 
again emphasize that for all values of $r$ our model, run on the posterior 
distributions of its parameters, reproduce the observed SMF of SF centrals 
and SFS.  
At $t_\mathrm{duty}{\geq}5\,\mathrm{Gyr}$ we find no significant difference 
in $\sigma_{\log\,M_*}$, regardless of $r$. Below $t_\mathrm{duty} < 5\,\mathrm{Gyr}$, 
however, $\sigma_{\log\,M_*}$ of our model decreases significantly with 
stronger assembly bias in our model. For $t_\mathrm{duty} = 0.5\,\mathrm{Gyr}$, 
{\color{red}
we find $\sigma_{\log\,M_*}{=}\,0.26\substack{+0.010\\-0.012}, 
0.22\substack{+0.013\\-0.005}$, and $0.17\substack{+0.007\\-0.015} $ 
}
for $r{=}0.0, 0.5$, and 
$0.99$, respectively. Comparing our updated model to the constraints included 
in Figure~\ref{fig:sigMstar_duty}, we find that incorporating assembly bias 
significantly reduces the tensions with observations (right panel). In fact, 
\emph{with a short star formation duty cycle ($t_\mathrm{duty} \leq 1\,\mathrm{Gyr}$) 
and strong assembly bias, our model conservatively reproduces 
$\sigma_{\log\,M_*}$ constraints from the literature.}

In addition to the observational constraints, we also compare the SHMR 
scatters predicted by our models to $\sigma_{\log\,M_*}$ from modern 
galaxy formation models on the right panel: hydrodynamic simulations 
(dot filled), semi-analytic models (hatched), and an empirical model 
(dashed line). For the hydrodynamic simulations, the dotted region 
encompasses constraints from EAGLE~\citep{mcalpine2016}, Massive 
Black II~\citep{khandai2015}, and Illustris TNG~\citep{pillepich2018}, 
see Figure 8 of \cite{wechsler2018}. For the semi-analytic models, the 
hatched region includes~\cite{lu2014, somerville2012}, 
and the SAGE model\footnote{\url{https://tao.asvo.org.au/tao/}}. Finally, we 
include the empirical model from \cite{behroozi2018a}, the {\sc UniverseMachine}.
By varying $r$ and $t_\mathrm{duty}$, $\sigma_{\log\,M_*}$ of our model 
encompasses all of the constraints from galaxy formation models. However, as
discussed in \cite{wechsler2018}, only the hydrodynamic simulations are in 
agreement with observations while semi-analytic models and {\sc UniverseMachine}
predict SHMR scatters that are too high. Meanwhile, our model, with short 
$t_\mathrm{duty}$ and high $r$, $\sigma_{\log\,M_*}$, is in good agreement with 
observations. 

A key element of our model is the SFH prescription for SF central galaxies where 
the SFH evolves about the SFS. Contrary to our SFH prescription, \cite{kelson2014}, 
for example, argue that the SFS is a simple consequence of central limit theorem 
and can be reproduced even if \emph{in situ} stellar mass growth is modeled as 
a stochastic process like a random walk. \cite{gladders2013,abramson2015,abramson2016}, 
similarly argue that $\sim2000$ loosely constrained log-normal SFHs can reproduce 
observations such as the SMF at $z \leq 8$ and the SFS at $z \leq 6$. These works, 
however, focus on reproducing observations of galaxy properties and do not examine
the galaxy-halo connection such as the SHMR. In order to test whether log-normal 
SFHs can also produce realistic SHMRs, we take the SFHs, $\mathrm{SFR}(t)$ and 
$M_*(t)$, from \cite{abramson2016} and assign them to halos by abundance matching 
their $M_*$ to $M_h$ at $z{\sim}1$. We then restrict the SFHs to those that would 
be classified as star-forming based on a rough $\log\,\mathrm{SSFR} > -11.$ cut.
Afterwards we measure $\sigma_{\log\,M_*}$ at the lowest $M_h$ where it can be 
reliably measured given the \cite{abramson2016} sample's $M_*{>}10^{10}M_\odot$ 
limit. Based o our abundance matching prescription, we find \cite{abramson2016}
predicts a scatter of $\sigma_{\log\,M_*}(M_h=10^{12.4}) = 0.33\pm0.04$ (dotted 
line in Figure~\ref{fig:sigMstar_duty_abias}). Despite the \cite{abramson2016} SFHs 
can reproduce a number of galaxy observations, $\sigma_{\log\,M_*}$ derived from 
the SFHs and abundance matching predict SHMR scatters too high compared to 
constraints from observations and predictions from hydrodynamic simulations. 
This estimate is, however, derived from a simple abundance matching scheme. 
As \cite{diemer2017} find from their log-normal fits to the SFHs of Illustris 
galaxies, halo formation history correlates with the fits. Incorporating such 
assembly bias into our abundance matching may reduce $\sigma_{\log\,M_*}$.

In this section, we demonstrate that incorporating assembly bias into our model
by correlating $\Delta\logsfr$ to $\Delta M_h$ reduces the tension with the 
SHMR scatter in observations. With assembly bias added, our model can produce 
the range of $\sigma_{\log\,M_*}$ constraints from observations as well as 
modern galaxy formation models. However, even with assembly bias a short duty 
cycle is \emph{necessary} to produce tight scatter in the SHMR. This further 
confirms our conclusion from the previous section that the SFRs of star-forming 
central galaxies have variations on $\lesssim 0.5\,\mathrm{Gyr}$ timescales.
Furthermore, with a short star formation duty cycle timescale and strong 
assembly bias $r > 0.5$, our model conservatively agrees with observational 
constraints and produces tight scatter in the SHMR, unlike predictions from
semi-analytic models and {\sc UniverseMachine}. Our comparison not only provides 
constraints on the timescale of star formation variability, but it also confirms
the role of assembly bias in the SFH of star-forming galaxies.

\section{Summary and Conclusion} \label{sec:summary}
\todo{some brief intro paragraph consistent with the intro} 
%Star forming galaxies are observed to have a tight relationship between their star  formation rates and stellar masses. This so-called ``star-forming sequence'' (SFS),  observed out to $z{\sim}2$, characterizes both the star formation history and stellar  mass growth of star forming galaxies. Based on observed constraints on the stellar-to-halo mass relation (SHMR), halo accretion history also likely plays  a role in the evolution of star forming galaxies. 

We combine the high-resolution cosmological $N$-body $\mathtt{TreePM}$ simulation 
with SFHs that evolve the SF central galaxies along the SFS and present a model 
that tracks the SFR, $M_*$, and host halo accretion histories of SF centrals from $z \sim 1$
to $z=0.05$. More specifically, we characterize the SFHs to evolve with respect to 
the mean $\log\,\mathrm{SFR}$ of the SFS with a ``star formation duty cycle'' that 
introduces variability on some specific timescale, $t_\mathrm{duty}$. Besides $t_\mathrm{duty}$,
we parameterize the SFS using \todo{parameters}, free parameters that dictate the 
low $M_*$ and high $M_*$ slopes and redshift evolution. With priors for these 
parameters set to only produce SFS within observed range, we use ABC-PMC to infer 
a model that reproduce the observed SMF of the SF centrals in the SDSS DR7 group 
catalog. When we examine the SHMR of this model we find: 
\bitem
\item A shorter star formation duty cycle in our model produces significantly 
tighter scatter in the SHMR both for fixed $M_h = 10^{12} M_\odot$ and 
$M_* = 10^{10}M_\odot$. As $t_\mathrm{duty}$ ranges from $10$ to $0.5\,\mathrm{Gyr}$, 
$\sigma_{\log\,M_*}{=}\,0.32\substack{+0.019\\ -0.021}$ to $0.26\substack{+0.010\\-0.012}$ 
and $\sigma_{\log\,M_h}{=}\,0.30\substack{+0.005\\ -0.006}$ to $0.19\substack{+0.008\\-0.006}$. 
The dependence of $\sigma_{\log\,M_*}$ and $\sigma_{\log\,M_h}$ on $t_\mathrm{duty}$ 
demonstrates that the scatter in SHMR provides some constraint on $t_\mathrm{duty}$ and
the timescale of star formation variability. 

\item We compare the $\sigma_{\log\,M_*}$ and $\sigma_{\log\,M_h}$ predicted by our 
model to constraints from halo occupation modeling of galaxy clustering, SMF, 
satellite kinematics, and galaxy-galaxy weak lensing observations. The $\sigma_{\log\,M_h}$
comparison does not significantly constrain $t_\mathrm{duty}$. However, from the 
$\sigma_{\log\,M_*}$ comparison, we find that a duty cycle with 
$t_\mathrm{duty} \lesssim 0.5\,\mathrm{Gyr}$ is necessary to reduce the tension 
with observations.

\item %In order to further reduce the tension with observations, 
We next incorporate 
assembly bias into our model by allowing the star formation histories to 
correlate with halo accretion histories by a correlation coefficient, $r$. 
With assembly bias, our model can predict $\sigma_{\log M_*}{=}0.17$ to $0.33$, 
which spans the predictions from modern galaxy formation models. %hydrodynamic simulations, semi-analytic models,  and the {\sc UniverseMachine} empirical model. 
However, to conservatively reproduce $\sigma_{\log M_*}$ predictions from 
hydrodynamic simulations and constraints from observations, a short duty 
cycle and $r > 0.5$--- \emph{i.e.} strong assembly bias--- is required. 
\eitem 

\todo{Paragraph about implications of our results} 
%The necessity of a short timescale duty cycle. 

\todo{Paragraph about future with DESI BGS.}


%By combining a high-resolution  cosmological $N$-body simulation with observed evolutionary trends of the SFS,  we present a model that tracks the star formation, stellar mass, and host halo  mass histories of star forming central galaxies over $z < 1$ and reproduces the observed stellar mass 
%function and SFS of central galaxies in the SDSS Data Release 7. We characterize
%the star formation variability in the galaxies using a ``star formation duty cycle'' 
%prescription, which fluctuates about the mean SFS on some timescale. A short duty 
%cycle $\lesssim 0.5\,\mathrm{Gyr}$ is \emph{necessary} for our model to produce a 
%tight scatter in the SHMR, $\sigma_{\log M_*}$, at $M_h{=}10^{12}M_\odot$ comparable 
%to observations. %\todo{sentence on the implication of a short duty cycle}.
%While a short duty cycle is necessary, to conservatively reproduce the observed 
%$\sigma_{\log M_*}$, the star formation histories must also correlate strongly 
%with halo accretion history --- \emph{i.e.} exhibit strong assembly bias. The 
%timescale of star formation variability and the correlation between star formation 
%and halo accretion history we infer, provide key constraints on star formation history 
%and the evolution of star forming galaxies.

%\appendix
%\section{$z \sim 1$ Initial Conditions} \label{app:z1}
%Much of the results presented in this paper are based on comparison 
%between our model and observations at $z \sim 0.$. Our model is initalized 
%at $z \sim 1$. Therefore, in this section we test some of the choices 
%we make in our intializations. 
%\bitem
%\item Test impact of $z \sim 1$ SMF
%\item Test impact of $z \sim 1$ $\sigma_{\log M_*}$ 
%\eitem

%%%%%%%%%%%%%%%%%%%%%%%%%%%%%%%%%%%%%%%%%%%%%%%%%%%%%%%%%%%%%%%
% Acknowledgements
%%%%%%%%%%%%%%%%%%%%%%%%%%%%%%%%%%%%%%%%%%%%%%%%%%%%%%%%%%%%%%%
\section*{Acknowledgements}
It's a pleasure to thank 
    J.D.~Cohn, 
    Shirley~Ho, 
    and 
    Tjitske~Starkenburg 
for valuable discussions and feedback. We also thank Louis E. Abramson, 
Junzhi Cao, Shy Genel, and Cheng Li for providing us with data used in 
the analysis. This material is based upon work supported by the U.S. 
Department of Energy, Office of Science, Office of High Energy Physics, 
under contract No. DE-AC02-05CH11231.

\bibliographystyle{aasjournal}
\bibliography{centralMS}
\end{document}
